\documentclass[11pt]{article}

    \usepackage[breakable]{tcolorbox}
    \usepackage{parskip} % Stop auto-indenting (to mimic markdown behaviour)
    

    % Basic figure setup, for now with no caption control since it's done
    % automatically by Pandoc (which extracts ![](path) syntax from Markdown).
    \usepackage{graphicx}
    % Keep aspect ratio if custom image width or height is specified
    \setkeys{Gin}{keepaspectratio}
    % Maintain compatibility with old templates. Remove in nbconvert 6.0
    \let\Oldincludegraphics\includegraphics
    % Ensure that by default, figures have no caption (until we provide a
    % proper Figure object with a Caption API and a way to capture that
    % in the conversion process - todo).
    \usepackage{caption}
    \DeclareCaptionFormat{nocaption}{}
    \captionsetup{format=nocaption,aboveskip=0pt,belowskip=0pt}

    \usepackage{float}
    \floatplacement{figure}{H} % forces figures to be placed at the correct location
    \usepackage{xcolor} % Allow colors to be defined
    \usepackage{enumerate} % Needed for markdown enumerations to work
    \usepackage{geometry} % Used to adjust the document margins
    \usepackage{amsmath} % Equations
    \usepackage{amssymb} % Equations
    \usepackage{textcomp} % defines textquotesingle
    % Hack from http://tex.stackexchange.com/a/47451/13684:
    \AtBeginDocument{%
        \def\PYZsq{\textquotesingle}% Upright quotes in Pygmentized code
    }
    \usepackage{upquote} % Upright quotes for verbatim code
    \usepackage{eurosym} % defines \euro

    \usepackage{iftex}
    \ifPDFTeX
        \usepackage[T1]{fontenc}
        \IfFileExists{alphabeta.sty}{
              \usepackage{alphabeta}
          }{
              \usepackage[mathletters]{ucs}
              \usepackage[utf8x]{inputenc}
          }
    \else
        \usepackage{fontspec}
        \usepackage{unicode-math}
    \fi

    \usepackage{fancyvrb} % verbatim replacement that allows latex
    \usepackage{grffile} % extends the file name processing of package graphics
                         % to support a larger range
    \makeatletter % fix for old versions of grffile with XeLaTeX
    \@ifpackagelater{grffile}{2019/11/01}
    {
      % Do nothing on new versions
    }
    {
      \def\Gread@@xetex#1{%
        \IfFileExists{"\Gin@base".bb}%
        {\Gread@eps{\Gin@base.bb}}%
        {\Gread@@xetex@aux#1}%
      }
    }
    \makeatother
    \usepackage[Export]{adjustbox} % Used to constrain images to a maximum size
    \adjustboxset{max size={0.9\linewidth}{0.9\paperheight}}

    % The hyperref package gives us a pdf with properly built
    % internal navigation ('pdf bookmarks' for the table of contents,
    % internal cross-reference links, web links for URLs, etc.)
    \usepackage{hyperref}
    % The default LaTeX title has an obnoxious amount of whitespace. By default,
    % titling removes some of it. It also provides customization options.
    \usepackage{titling}
    \usepackage{longtable} % longtable support required by pandoc >1.10
    \usepackage{booktabs}  % table support for pandoc > 1.12.2
    \usepackage{array}     % table support for pandoc >= 2.11.3
    \usepackage{calc}      % table minipage width calculation for pandoc >= 2.11.1
    \usepackage[inline]{enumitem} % IRkernel/repr support (it uses the enumerate* environment)
    \usepackage[normalem]{ulem} % ulem is needed to support strikethroughs (\sout)
                                % normalem makes italics be italics, not underlines
    \usepackage{soul}      % strikethrough (\st) support for pandoc >= 3.0.0
    \usepackage{mathrsfs}
    

    
    % Colors for the hyperref package
    \definecolor{urlcolor}{rgb}{0,.145,.698}
    \definecolor{linkcolor}{rgb}{.71,0.21,0.01}
    \definecolor{citecolor}{rgb}{.12,.54,.11}

    % ANSI colors
    \definecolor{ansi-black}{HTML}{3E424D}
    \definecolor{ansi-black-intense}{HTML}{282C36}
    \definecolor{ansi-red}{HTML}{E75C58}
    \definecolor{ansi-red-intense}{HTML}{B22B31}
    \definecolor{ansi-green}{HTML}{00A250}
    \definecolor{ansi-green-intense}{HTML}{007427}
    \definecolor{ansi-yellow}{HTML}{DDB62B}
    \definecolor{ansi-yellow-intense}{HTML}{B27D12}
    \definecolor{ansi-blue}{HTML}{208FFB}
    \definecolor{ansi-blue-intense}{HTML}{0065CA}
    \definecolor{ansi-magenta}{HTML}{D160C4}
    \definecolor{ansi-magenta-intense}{HTML}{A03196}
    \definecolor{ansi-cyan}{HTML}{60C6C8}
    \definecolor{ansi-cyan-intense}{HTML}{258F8F}
    \definecolor{ansi-white}{HTML}{C5C1B4}
    \definecolor{ansi-white-intense}{HTML}{A1A6B2}
    \definecolor{ansi-default-inverse-fg}{HTML}{FFFFFF}
    \definecolor{ansi-default-inverse-bg}{HTML}{000000}

    % common color for the border for error outputs.
    \definecolor{outerrorbackground}{HTML}{FFDFDF}

    % commands and environments needed by pandoc snippets
    % extracted from the output of `pandoc -s`
    \providecommand{\tightlist}{%
      \setlength{\itemsep}{0pt}\setlength{\parskip}{0pt}}
    \DefineVerbatimEnvironment{Highlighting}{Verbatim}{commandchars=\\\{\}}
    % Add ',fontsize=\small' for more characters per line
    \newenvironment{Shaded}{}{}
    \newcommand{\KeywordTok}[1]{\textcolor[rgb]{0.00,0.44,0.13}{\textbf{{#1}}}}
    \newcommand{\DataTypeTok}[1]{\textcolor[rgb]{0.56,0.13,0.00}{{#1}}}
    \newcommand{\DecValTok}[1]{\textcolor[rgb]{0.25,0.63,0.44}{{#1}}}
    \newcommand{\BaseNTok}[1]{\textcolor[rgb]{0.25,0.63,0.44}{{#1}}}
    \newcommand{\FloatTok}[1]{\textcolor[rgb]{0.25,0.63,0.44}{{#1}}}
    \newcommand{\CharTok}[1]{\textcolor[rgb]{0.25,0.44,0.63}{{#1}}}
    \newcommand{\StringTok}[1]{\textcolor[rgb]{0.25,0.44,0.63}{{#1}}}
    \newcommand{\CommentTok}[1]{\textcolor[rgb]{0.38,0.63,0.69}{\textit{{#1}}}}
    \newcommand{\OtherTok}[1]{\textcolor[rgb]{0.00,0.44,0.13}{{#1}}}
    \newcommand{\AlertTok}[1]{\textcolor[rgb]{1.00,0.00,0.00}{\textbf{{#1}}}}
    \newcommand{\FunctionTok}[1]{\textcolor[rgb]{0.02,0.16,0.49}{{#1}}}
    \newcommand{\RegionMarkerTok}[1]{{#1}}
    \newcommand{\ErrorTok}[1]{\textcolor[rgb]{1.00,0.00,0.00}{\textbf{{#1}}}}
    \newcommand{\NormalTok}[1]{{#1}}

    % Additional commands for more recent versions of Pandoc
    \newcommand{\ConstantTok}[1]{\textcolor[rgb]{0.53,0.00,0.00}{{#1}}}
    \newcommand{\SpecialCharTok}[1]{\textcolor[rgb]{0.25,0.44,0.63}{{#1}}}
    \newcommand{\VerbatimStringTok}[1]{\textcolor[rgb]{0.25,0.44,0.63}{{#1}}}
    \newcommand{\SpecialStringTok}[1]{\textcolor[rgb]{0.73,0.40,0.53}{{#1}}}
    \newcommand{\ImportTok}[1]{{#1}}
    \newcommand{\DocumentationTok}[1]{\textcolor[rgb]{0.73,0.13,0.13}{\textit{{#1}}}}
    \newcommand{\AnnotationTok}[1]{\textcolor[rgb]{0.38,0.63,0.69}{\textbf{\textit{{#1}}}}}
    \newcommand{\CommentVarTok}[1]{\textcolor[rgb]{0.38,0.63,0.69}{\textbf{\textit{{#1}}}}}
    \newcommand{\VariableTok}[1]{\textcolor[rgb]{0.10,0.09,0.49}{{#1}}}
    \newcommand{\ControlFlowTok}[1]{\textcolor[rgb]{0.00,0.44,0.13}{\textbf{{#1}}}}
    \newcommand{\OperatorTok}[1]{\textcolor[rgb]{0.40,0.40,0.40}{{#1}}}
    \newcommand{\BuiltInTok}[1]{{#1}}
    \newcommand{\ExtensionTok}[1]{{#1}}
    \newcommand{\PreprocessorTok}[1]{\textcolor[rgb]{0.74,0.48,0.00}{{#1}}}
    \newcommand{\AttributeTok}[1]{\textcolor[rgb]{0.49,0.56,0.16}{{#1}}}
    \newcommand{\InformationTok}[1]{\textcolor[rgb]{0.38,0.63,0.69}{\textbf{\textit{{#1}}}}}
    \newcommand{\WarningTok}[1]{\textcolor[rgb]{0.38,0.63,0.69}{\textbf{\textit{{#1}}}}}
    \makeatletter
    \newsavebox\pandoc@box
    \newcommand*\pandocbounded[1]{%
      \sbox\pandoc@box{#1}%
      % scaling factors for width and height
      \Gscale@div\@tempa\textheight{\dimexpr\ht\pandoc@box+\dp\pandoc@box\relax}%
      \Gscale@div\@tempb\linewidth{\wd\pandoc@box}%
      % select the smaller of both
      \ifdim\@tempb\p@<\@tempa\p@
        \let\@tempa\@tempb
      \fi
      % scaling accordingly (\@tempa < 1)
      \ifdim\@tempa\p@<\p@
        \scalebox{\@tempa}{\usebox\pandoc@box}%
      % scaling not needed, use as it is
      \else
        \usebox{\pandoc@box}%
      \fi
    }
    \makeatother

    % Define a nice break command that doesn't care if a line doesn't already
    % exist.
    \def\br{\hspace*{\fill} \\* }
    % Math Jax compatibility definitions
    \def\gt{>}
    \def\lt{<}
    \let\Oldtex\TeX
    \let\Oldlatex\LaTeX
    \renewcommand{\TeX}{\textrm{\Oldtex}}
    \renewcommand{\LaTeX}{\textrm{\Oldlatex}}
    % Document parameters
    % Document title
    \title{dotnet}
    
    
    
    
    
    
    
% Pygments definitions
\makeatletter
\def\PY@reset{\let\PY@it=\relax \let\PY@bf=\relax%
    \let\PY@ul=\relax \let\PY@tc=\relax%
    \let\PY@bc=\relax \let\PY@ff=\relax}
\def\PY@tok#1{\csname PY@tok@#1\endcsname}
\def\PY@toks#1+{\ifx\relax#1\empty\else%
    \PY@tok{#1}\expandafter\PY@toks\fi}
\def\PY@do#1{\PY@bc{\PY@tc{\PY@ul{%
    \PY@it{\PY@bf{\PY@ff{#1}}}}}}}
\def\PY#1#2{\PY@reset\PY@toks#1+\relax+\PY@do{#2}}

\@namedef{PY@tok@w}{\def\PY@tc##1{\textcolor[rgb]{0.73,0.73,0.73}{##1}}}
\@namedef{PY@tok@c}{\let\PY@it=\textit\def\PY@tc##1{\textcolor[rgb]{0.24,0.48,0.48}{##1}}}
\@namedef{PY@tok@cp}{\def\PY@tc##1{\textcolor[rgb]{0.61,0.40,0.00}{##1}}}
\@namedef{PY@tok@k}{\let\PY@bf=\textbf\def\PY@tc##1{\textcolor[rgb]{0.00,0.50,0.00}{##1}}}
\@namedef{PY@tok@kp}{\def\PY@tc##1{\textcolor[rgb]{0.00,0.50,0.00}{##1}}}
\@namedef{PY@tok@kt}{\def\PY@tc##1{\textcolor[rgb]{0.69,0.00,0.25}{##1}}}
\@namedef{PY@tok@o}{\def\PY@tc##1{\textcolor[rgb]{0.40,0.40,0.40}{##1}}}
\@namedef{PY@tok@ow}{\let\PY@bf=\textbf\def\PY@tc##1{\textcolor[rgb]{0.67,0.13,1.00}{##1}}}
\@namedef{PY@tok@nb}{\def\PY@tc##1{\textcolor[rgb]{0.00,0.50,0.00}{##1}}}
\@namedef{PY@tok@nf}{\def\PY@tc##1{\textcolor[rgb]{0.00,0.00,1.00}{##1}}}
\@namedef{PY@tok@nc}{\let\PY@bf=\textbf\def\PY@tc##1{\textcolor[rgb]{0.00,0.00,1.00}{##1}}}
\@namedef{PY@tok@nn}{\let\PY@bf=\textbf\def\PY@tc##1{\textcolor[rgb]{0.00,0.00,1.00}{##1}}}
\@namedef{PY@tok@ne}{\let\PY@bf=\textbf\def\PY@tc##1{\textcolor[rgb]{0.80,0.25,0.22}{##1}}}
\@namedef{PY@tok@nv}{\def\PY@tc##1{\textcolor[rgb]{0.10,0.09,0.49}{##1}}}
\@namedef{PY@tok@no}{\def\PY@tc##1{\textcolor[rgb]{0.53,0.00,0.00}{##1}}}
\@namedef{PY@tok@nl}{\def\PY@tc##1{\textcolor[rgb]{0.46,0.46,0.00}{##1}}}
\@namedef{PY@tok@ni}{\let\PY@bf=\textbf\def\PY@tc##1{\textcolor[rgb]{0.44,0.44,0.44}{##1}}}
\@namedef{PY@tok@na}{\def\PY@tc##1{\textcolor[rgb]{0.41,0.47,0.13}{##1}}}
\@namedef{PY@tok@nt}{\let\PY@bf=\textbf\def\PY@tc##1{\textcolor[rgb]{0.00,0.50,0.00}{##1}}}
\@namedef{PY@tok@nd}{\def\PY@tc##1{\textcolor[rgb]{0.67,0.13,1.00}{##1}}}
\@namedef{PY@tok@s}{\def\PY@tc##1{\textcolor[rgb]{0.73,0.13,0.13}{##1}}}
\@namedef{PY@tok@sd}{\let\PY@it=\textit\def\PY@tc##1{\textcolor[rgb]{0.73,0.13,0.13}{##1}}}
\@namedef{PY@tok@si}{\let\PY@bf=\textbf\def\PY@tc##1{\textcolor[rgb]{0.64,0.35,0.47}{##1}}}
\@namedef{PY@tok@se}{\let\PY@bf=\textbf\def\PY@tc##1{\textcolor[rgb]{0.67,0.36,0.12}{##1}}}
\@namedef{PY@tok@sr}{\def\PY@tc##1{\textcolor[rgb]{0.64,0.35,0.47}{##1}}}
\@namedef{PY@tok@ss}{\def\PY@tc##1{\textcolor[rgb]{0.10,0.09,0.49}{##1}}}
\@namedef{PY@tok@sx}{\def\PY@tc##1{\textcolor[rgb]{0.00,0.50,0.00}{##1}}}
\@namedef{PY@tok@m}{\def\PY@tc##1{\textcolor[rgb]{0.40,0.40,0.40}{##1}}}
\@namedef{PY@tok@gh}{\let\PY@bf=\textbf\def\PY@tc##1{\textcolor[rgb]{0.00,0.00,0.50}{##1}}}
\@namedef{PY@tok@gu}{\let\PY@bf=\textbf\def\PY@tc##1{\textcolor[rgb]{0.50,0.00,0.50}{##1}}}
\@namedef{PY@tok@gd}{\def\PY@tc##1{\textcolor[rgb]{0.63,0.00,0.00}{##1}}}
\@namedef{PY@tok@gi}{\def\PY@tc##1{\textcolor[rgb]{0.00,0.52,0.00}{##1}}}
\@namedef{PY@tok@gr}{\def\PY@tc##1{\textcolor[rgb]{0.89,0.00,0.00}{##1}}}
\@namedef{PY@tok@ge}{\let\PY@it=\textit}
\@namedef{PY@tok@gs}{\let\PY@bf=\textbf}
\@namedef{PY@tok@ges}{\let\PY@bf=\textbf\let\PY@it=\textit}
\@namedef{PY@tok@gp}{\let\PY@bf=\textbf\def\PY@tc##1{\textcolor[rgb]{0.00,0.00,0.50}{##1}}}
\@namedef{PY@tok@go}{\def\PY@tc##1{\textcolor[rgb]{0.44,0.44,0.44}{##1}}}
\@namedef{PY@tok@gt}{\def\PY@tc##1{\textcolor[rgb]{0.00,0.27,0.87}{##1}}}
\@namedef{PY@tok@err}{\def\PY@bc##1{{\setlength{\fboxsep}{\string -\fboxrule}\fcolorbox[rgb]{1.00,0.00,0.00}{1,1,1}{\strut ##1}}}}
\@namedef{PY@tok@kc}{\let\PY@bf=\textbf\def\PY@tc##1{\textcolor[rgb]{0.00,0.50,0.00}{##1}}}
\@namedef{PY@tok@kd}{\let\PY@bf=\textbf\def\PY@tc##1{\textcolor[rgb]{0.00,0.50,0.00}{##1}}}
\@namedef{PY@tok@kn}{\let\PY@bf=\textbf\def\PY@tc##1{\textcolor[rgb]{0.00,0.50,0.00}{##1}}}
\@namedef{PY@tok@kr}{\let\PY@bf=\textbf\def\PY@tc##1{\textcolor[rgb]{0.00,0.50,0.00}{##1}}}
\@namedef{PY@tok@bp}{\def\PY@tc##1{\textcolor[rgb]{0.00,0.50,0.00}{##1}}}
\@namedef{PY@tok@fm}{\def\PY@tc##1{\textcolor[rgb]{0.00,0.00,1.00}{##1}}}
\@namedef{PY@tok@vc}{\def\PY@tc##1{\textcolor[rgb]{0.10,0.09,0.49}{##1}}}
\@namedef{PY@tok@vg}{\def\PY@tc##1{\textcolor[rgb]{0.10,0.09,0.49}{##1}}}
\@namedef{PY@tok@vi}{\def\PY@tc##1{\textcolor[rgb]{0.10,0.09,0.49}{##1}}}
\@namedef{PY@tok@vm}{\def\PY@tc##1{\textcolor[rgb]{0.10,0.09,0.49}{##1}}}
\@namedef{PY@tok@sa}{\def\PY@tc##1{\textcolor[rgb]{0.73,0.13,0.13}{##1}}}
\@namedef{PY@tok@sb}{\def\PY@tc##1{\textcolor[rgb]{0.73,0.13,0.13}{##1}}}
\@namedef{PY@tok@sc}{\def\PY@tc##1{\textcolor[rgb]{0.73,0.13,0.13}{##1}}}
\@namedef{PY@tok@dl}{\def\PY@tc##1{\textcolor[rgb]{0.73,0.13,0.13}{##1}}}
\@namedef{PY@tok@s2}{\def\PY@tc##1{\textcolor[rgb]{0.73,0.13,0.13}{##1}}}
\@namedef{PY@tok@sh}{\def\PY@tc##1{\textcolor[rgb]{0.73,0.13,0.13}{##1}}}
\@namedef{PY@tok@s1}{\def\PY@tc##1{\textcolor[rgb]{0.73,0.13,0.13}{##1}}}
\@namedef{PY@tok@mb}{\def\PY@tc##1{\textcolor[rgb]{0.40,0.40,0.40}{##1}}}
\@namedef{PY@tok@mf}{\def\PY@tc##1{\textcolor[rgb]{0.40,0.40,0.40}{##1}}}
\@namedef{PY@tok@mh}{\def\PY@tc##1{\textcolor[rgb]{0.40,0.40,0.40}{##1}}}
\@namedef{PY@tok@mi}{\def\PY@tc##1{\textcolor[rgb]{0.40,0.40,0.40}{##1}}}
\@namedef{PY@tok@il}{\def\PY@tc##1{\textcolor[rgb]{0.40,0.40,0.40}{##1}}}
\@namedef{PY@tok@mo}{\def\PY@tc##1{\textcolor[rgb]{0.40,0.40,0.40}{##1}}}
\@namedef{PY@tok@ch}{\let\PY@it=\textit\def\PY@tc##1{\textcolor[rgb]{0.24,0.48,0.48}{##1}}}
\@namedef{PY@tok@cm}{\let\PY@it=\textit\def\PY@tc##1{\textcolor[rgb]{0.24,0.48,0.48}{##1}}}
\@namedef{PY@tok@cpf}{\let\PY@it=\textit\def\PY@tc##1{\textcolor[rgb]{0.24,0.48,0.48}{##1}}}
\@namedef{PY@tok@c1}{\let\PY@it=\textit\def\PY@tc##1{\textcolor[rgb]{0.24,0.48,0.48}{##1}}}
\@namedef{PY@tok@cs}{\let\PY@it=\textit\def\PY@tc##1{\textcolor[rgb]{0.24,0.48,0.48}{##1}}}

\def\PYZbs{\char`\\}
\def\PYZus{\char`\_}
\def\PYZob{\char`\{}
\def\PYZcb{\char`\}}
\def\PYZca{\char`\^}
\def\PYZam{\char`\&}
\def\PYZlt{\char`\<}
\def\PYZgt{\char`\>}
\def\PYZsh{\char`\#}
\def\PYZpc{\char`\%}
\def\PYZdl{\char`\$}
\def\PYZhy{\char`\-}
\def\PYZsq{\char`\'}
\def\PYZdq{\char`\"}
\def\PYZti{\char`\~}
% for compatibility with earlier versions
\def\PYZat{@}
\def\PYZlb{[}
\def\PYZrb{]}
\makeatother


    % For linebreaks inside Verbatim environment from package fancyvrb.
    \makeatletter
        \newbox\Wrappedcontinuationbox
        \newbox\Wrappedvisiblespacebox
        \newcommand*\Wrappedvisiblespace {\textcolor{red}{\textvisiblespace}}
        \newcommand*\Wrappedcontinuationsymbol {\textcolor{red}{\llap{\tiny$\m@th\hookrightarrow$}}}
        \newcommand*\Wrappedcontinuationindent {3ex }
        \newcommand*\Wrappedafterbreak {\kern\Wrappedcontinuationindent\copy\Wrappedcontinuationbox}
        % Take advantage of the already applied Pygments mark-up to insert
        % potential linebreaks for TeX processing.
        %        {, <, #, %, $, ' and ": go to next line.
        %        _, }, ^, &, >, - and ~: stay at end of broken line.
        % Use of \textquotesingle for straight quote.
        \newcommand*\Wrappedbreaksatspecials {%
            \def\PYGZus{\discretionary{\char`\_}{\Wrappedafterbreak}{\char`\_}}%
            \def\PYGZob{\discretionary{}{\Wrappedafterbreak\char`\{}{\char`\{}}%
            \def\PYGZcb{\discretionary{\char`\}}{\Wrappedafterbreak}{\char`\}}}%
            \def\PYGZca{\discretionary{\char`\^}{\Wrappedafterbreak}{\char`\^}}%
            \def\PYGZam{\discretionary{\char`\&}{\Wrappedafterbreak}{\char`\&}}%
            \def\PYGZlt{\discretionary{}{\Wrappedafterbreak\char`\<}{\char`\<}}%
            \def\PYGZgt{\discretionary{\char`\>}{\Wrappedafterbreak}{\char`\>}}%
            \def\PYGZsh{\discretionary{}{\Wrappedafterbreak\char`\#}{\char`\#}}%
            \def\PYGZpc{\discretionary{}{\Wrappedafterbreak\char`\%}{\char`\%}}%
            \def\PYGZdl{\discretionary{}{\Wrappedafterbreak\char`\$}{\char`\$}}%
            \def\PYGZhy{\discretionary{\char`\-}{\Wrappedafterbreak}{\char`\-}}%
            \def\PYGZsq{\discretionary{}{\Wrappedafterbreak\textquotesingle}{\textquotesingle}}%
            \def\PYGZdq{\discretionary{}{\Wrappedafterbreak\char`\"}{\char`\"}}%
            \def\PYGZti{\discretionary{\char`\~}{\Wrappedafterbreak}{\char`\~}}%
        }
        % Some characters . , ; ? ! / are not pygmentized.
        % This macro makes them "active" and they will insert potential linebreaks
        \newcommand*\Wrappedbreaksatpunct {%
            \lccode`\~`\.\lowercase{\def~}{\discretionary{\hbox{\char`\.}}{\Wrappedafterbreak}{\hbox{\char`\.}}}%
            \lccode`\~`\,\lowercase{\def~}{\discretionary{\hbox{\char`\,}}{\Wrappedafterbreak}{\hbox{\char`\,}}}%
            \lccode`\~`\;\lowercase{\def~}{\discretionary{\hbox{\char`\;}}{\Wrappedafterbreak}{\hbox{\char`\;}}}%
            \lccode`\~`\:\lowercase{\def~}{\discretionary{\hbox{\char`\:}}{\Wrappedafterbreak}{\hbox{\char`\:}}}%
            \lccode`\~`\?\lowercase{\def~}{\discretionary{\hbox{\char`\?}}{\Wrappedafterbreak}{\hbox{\char`\?}}}%
            \lccode`\~`\!\lowercase{\def~}{\discretionary{\hbox{\char`\!}}{\Wrappedafterbreak}{\hbox{\char`\!}}}%
            \lccode`\~`\/\lowercase{\def~}{\discretionary{\hbox{\char`\/}}{\Wrappedafterbreak}{\hbox{\char`\/}}}%
            \catcode`\.\active
            \catcode`\,\active
            \catcode`\;\active
            \catcode`\:\active
            \catcode`\?\active
            \catcode`\!\active
            \catcode`\/\active
            \lccode`\~`\~
        }
    \makeatother

    \let\OriginalVerbatim=\Verbatim
    \makeatletter
    \renewcommand{\Verbatim}[1][1]{%
        %\parskip\z@skip
        \sbox\Wrappedcontinuationbox {\Wrappedcontinuationsymbol}%
        \sbox\Wrappedvisiblespacebox {\FV@SetupFont\Wrappedvisiblespace}%
        \def\FancyVerbFormatLine ##1{\hsize\linewidth
            \vtop{\raggedright\hyphenpenalty\z@\exhyphenpenalty\z@
                \doublehyphendemerits\z@\finalhyphendemerits\z@
                \strut ##1\strut}%
        }%
        % If the linebreak is at a space, the latter will be displayed as visible
        % space at end of first line, and a continuation symbol starts next line.
        % Stretch/shrink are however usually zero for typewriter font.
        \def\FV@Space {%
            \nobreak\hskip\z@ plus\fontdimen3\font minus\fontdimen4\font
            \discretionary{\copy\Wrappedvisiblespacebox}{\Wrappedafterbreak}
            {\kern\fontdimen2\font}%
        }%

        % Allow breaks at special characters using \PYG... macros.
        \Wrappedbreaksatspecials
        % Breaks at punctuation characters . , ; ? ! and / need catcode=\active
        \OriginalVerbatim[#1,codes*=\Wrappedbreaksatpunct]%
    }
    \makeatother

    % Exact colors from NB
    \definecolor{incolor}{HTML}{303F9F}
    \definecolor{outcolor}{HTML}{D84315}
    \definecolor{cellborder}{HTML}{CFCFCF}
    \definecolor{cellbackground}{HTML}{F7F7F7}

    % prompt
    \makeatletter
    \newcommand{\boxspacing}{\kern\kvtcb@left@rule\kern\kvtcb@boxsep}
    \makeatother
    \newcommand{\prompt}[4]{
        {\ttfamily\llap{{\color{#2}[#3]:\hspace{3pt}#4}}\vspace{-\baselineskip}}
    }
    

    
    % Prevent overflowing lines due to hard-to-break entities
    \sloppy
    % Setup hyperref package
    \hypersetup{
      breaklinks=true,  % so long urls are correctly broken across lines
      colorlinks=true,
      urlcolor=urlcolor,
      linkcolor=linkcolor,
      citecolor=citecolor,
      }
    % Slightly bigger margins than the latex defaults
    
    \geometry{verbose,tmargin=1in,bmargin=1in,lmargin=1in,rmargin=1in}
    
    

\begin{document}
    
    \maketitle
    
    

    
    \hypertarget{write-a-c-program-to-demonstrate-five-types-of-constructor-in-c.}{%
\section{1. Write a C\# program to demonstrate five types of constructor
in
C\#.}\label{write-a-c-program-to-demonstrate-five-types-of-constructor-in-c.}}

Theory: A constructor is a special member of a class that is
automatically called when an object is created. It is mainly used to
initialize data members of a class. C\# supports different types of
constructors such as: Default Constructor -- Initializes object with
default values. Parameterized Constructor -- Accepts parameters to
initialize data members. Copy Constructor-- Copies values from one
object to another. Static Constructor -- Initializes static members of
the class and runs only once. Private Constructor -- Restricts object
creation from outside the class.

    \begin{tcolorbox}[breakable, size=fbox, boxrule=1pt, pad at break*=1mm,colback=cellbackground, colframe=cellborder]
\prompt{In}{incolor}{7}{\boxspacing}
\begin{Verbatim}[commandchars=\\\{\}]
\PY{c+c1}{// LAb 1}

\PY{k}{using}\PY{+w}{ }\PY{n+nn}{System}\PY{p}{;}

\PY{k}{class}\PY{+w}{ }\PY{n+nc}{Student}
\PY{p}{\PYZob{}}
\PY{+w}{    }\PY{k}{public}\PY{+w}{ }\PY{k+kt}{int}\PY{+w}{ }\PY{n}{Id}\PY{p}{;}
\PY{+w}{    }\PY{k}{public}\PY{+w}{ }\PY{k+kt}{string}\PY{+w}{ }\PY{n}{Name}\PY{p}{;}
\PY{+w}{    }\PY{k}{public}\PY{+w}{ }\PY{k}{static}\PY{+w}{ }\PY{k+kt}{string}\PY{+w}{ }\PY{n}{College}\PY{p}{;}

\PY{+w}{    }\PY{k}{static}\PY{+w}{ }\PY{n+nf}{Student}\PY{p}{(}\PY{p}{)}
\PY{+w}{    }\PY{p}{\PYZob{}}
\PY{+w}{        }\PY{n}{College}\PY{+w}{ }\PY{o}{=}\PY{+w}{ }\PY{l+s}{\PYZdq{}ABC Engineering College\PYZdq{}}\PY{p}{;}
\PY{+w}{        }\PY{n}{Console}\PY{p}{.}\PY{n}{WriteLine}\PY{p}{(}\PY{l+s}{\PYZdq{}Static Constructor called\PYZdq{}}\PY{p}{)}\PY{p}{;}
\PY{+w}{    }\PY{p}{\PYZcb{}}

\PY{+w}{    }\PY{k}{public}\PY{+w}{ }\PY{n+nf}{Student}\PY{p}{(}\PY{p}{)}
\PY{+w}{    }\PY{p}{\PYZob{}}
\PY{+w}{        }\PY{n}{Id}\PY{+w}{ }\PY{o}{=}\PY{+w}{ }\PY{l+m+mi}{0}\PY{p}{;}
\PY{+w}{        }\PY{n}{Name}\PY{+w}{ }\PY{o}{=}\PY{+w}{ }\PY{l+s}{\PYZdq{}Not Assigned\PYZdq{}}\PY{p}{;}
\PY{+w}{        }\PY{n}{Console}\PY{p}{.}\PY{n}{WriteLine}\PY{p}{(}\PY{l+s}{\PYZdq{}Default Constructor called\PYZdq{}}\PY{p}{)}\PY{p}{;}
\PY{+w}{    }\PY{p}{\PYZcb{}}

\PY{+w}{    }\PY{k}{public}\PY{+w}{ }\PY{n+nf}{Student}\PY{p}{(}\PY{k+kt}{int}\PY{+w}{ }\PY{n}{id}\PY{p}{,}\PY{+w}{ }\PY{k+kt}{string}\PY{+w}{ }\PY{n}{name}\PY{p}{)}
\PY{+w}{    }\PY{p}{\PYZob{}}
\PY{+w}{        }\PY{n}{Id}\PY{+w}{ }\PY{o}{=}\PY{+w}{ }\PY{n}{id}\PY{p}{;}
\PY{+w}{        }\PY{n}{Name}\PY{+w}{ }\PY{o}{=}\PY{+w}{ }\PY{n}{name}\PY{p}{;}
\PY{+w}{        }\PY{n}{Console}\PY{p}{.}\PY{n}{WriteLine}\PY{p}{(}\PY{l+s}{\PYZdq{}Parameterized Constructor called\PYZdq{}}\PY{p}{)}\PY{p}{;}
\PY{+w}{    }\PY{p}{\PYZcb{}}

\PY{+w}{    }\PY{k}{public}\PY{+w}{ }\PY{n+nf}{Student}\PY{p}{(}\PY{n}{Student}\PY{+w}{ }\PY{n}{s}\PY{p}{)}
\PY{+w}{    }\PY{p}{\PYZob{}}
\PY{+w}{        }\PY{n}{Id}\PY{+w}{ }\PY{o}{=}\PY{+w}{ }\PY{n}{s}\PY{p}{.}\PY{n}{Id}\PY{p}{;}
\PY{+w}{        }\PY{n}{Name}\PY{+w}{ }\PY{o}{=}\PY{+w}{ }\PY{n}{s}\PY{p}{.}\PY{n}{Name}\PY{p}{;}
\PY{+w}{        }\PY{n}{Console}\PY{p}{.}\PY{n}{WriteLine}\PY{p}{(}\PY{l+s}{\PYZdq{}Copy Constructor called\PYZdq{}}\PY{p}{)}\PY{p}{;}
\PY{+w}{    }\PY{p}{\PYZcb{}}

\PY{+w}{    }\PY{k}{public}\PY{+w}{ }\PY{k}{void}\PY{+w}{ }\PY{n+nf}{Display}\PY{p}{(}\PY{p}{)}
\PY{+w}{    }\PY{p}{\PYZob{}}
\PY{+w}{        }\PY{n}{Console}\PY{p}{.}\PY{n}{WriteLine}\PY{p}{(}\PY{l+s}{\PYZdq{}Id: \PYZdq{}}\PY{+w}{ }\PY{o}{+}\PY{+w}{ }\PY{n}{Id}\PY{+w}{ }\PY{o}{+}\PY{+w}{ }\PY{l+s}{\PYZdq{}, Name: \PYZdq{}}\PY{+w}{ }\PY{o}{+}\PY{+w}{ }\PY{n}{Name}\PY{+w}{ }\PY{o}{+}\PY{+w}{ }\PY{l+s}{\PYZdq{}, College: \PYZdq{}}\PY{+w}{ }\PY{o}{+}\PY{+w}{ }\PY{n}{College}\PY{p}{)}\PY{p}{;}
\PY{+w}{    }\PY{p}{\PYZcb{}}
\PY{p}{\PYZcb{}}

\PY{k}{class}\PY{+w}{ }\PY{n+nc}{Singleton}
\PY{p}{\PYZob{}}
\PY{+w}{    }\PY{k}{private}\PY{+w}{ }\PY{k}{static}\PY{+w}{ }\PY{n}{Singleton}\PY{+w}{ }\PY{n}{instance}\PY{p}{;}

\PY{+w}{    }\PY{k}{private}\PY{+w}{ }\PY{n+nf}{Singleton}\PY{p}{(}\PY{p}{)}
\PY{+w}{    }\PY{p}{\PYZob{}}
\PY{+w}{        }\PY{n}{Console}\PY{p}{.}\PY{n}{WriteLine}\PY{p}{(}\PY{l+s}{\PYZdq{}Private Constructor called\PYZdq{}}\PY{p}{)}\PY{p}{;}
\PY{+w}{    }\PY{p}{\PYZcb{}}

\PY{+w}{    }\PY{k}{public}\PY{+w}{ }\PY{k}{static}\PY{+w}{ }\PY{n}{Singleton}\PY{+w}{ }\PY{n+nf}{GetInstance}\PY{p}{(}\PY{p}{)}
\PY{+w}{    }\PY{p}{\PYZob{}}
\PY{+w}{        }\PY{k}{if}\PY{+w}{ }\PY{p}{(}\PY{n}{instance}\PY{+w}{ }\PY{o}{==}\PY{+w}{ }\PY{k}{null}\PY{p}{)}
\PY{+w}{            }\PY{n}{instance}\PY{+w}{ }\PY{o}{=}\PY{+w}{ }\PY{k}{new}\PY{+w}{ }\PY{n}{Singleton}\PY{p}{(}\PY{p}{)}\PY{p}{;}
\PY{+w}{        }\PY{k}{return}\PY{+w}{ }\PY{n}{instance}\PY{p}{;}
\PY{+w}{    }\PY{p}{\PYZcb{}}
\PY{p}{\PYZcb{}}

\PY{c+c1}{// Top\PYZhy{}level code}
\PY{n}{Student}\PY{+w}{ }\PY{n}{s1}\PY{+w}{ }\PY{o}{=}\PY{+w}{ }\PY{k}{new}\PY{+w}{ }\PY{n}{Student}\PY{p}{(}\PY{p}{)}\PY{p}{;}
\PY{n}{s1}\PY{p}{.}\PY{n}{Display}\PY{p}{(}\PY{p}{)}\PY{p}{;}

\PY{n}{Student}\PY{+w}{ }\PY{n}{s2}\PY{+w}{ }\PY{o}{=}\PY{+w}{ }\PY{k}{new}\PY{+w}{ }\PY{n}{Student}\PY{p}{(}\PY{l+m+mi}{101}\PY{p}{,}\PY{+w}{ }\PY{l+s}{\PYZdq{}Rahul\PYZdq{}}\PY{p}{)}\PY{p}{;}
\PY{n}{s2}\PY{p}{.}\PY{n}{Display}\PY{p}{(}\PY{p}{)}\PY{p}{;}

\PY{n}{Student}\PY{+w}{ }\PY{n}{s3}\PY{+w}{ }\PY{o}{=}\PY{+w}{ }\PY{k}{new}\PY{+w}{ }\PY{n}{Student}\PY{p}{(}\PY{n}{s2}\PY{p}{)}\PY{p}{;}
\PY{n}{s3}\PY{p}{.}\PY{n}{Display}\PY{p}{(}\PY{p}{)}\PY{p}{;}

\PY{n}{Singleton}\PY{+w}{ }\PY{n}{obj1}\PY{+w}{ }\PY{o}{=}\PY{+w}{ }\PY{n}{Singleton}\PY{p}{.}\PY{n}{GetInstance}\PY{p}{(}\PY{p}{)}\PY{p}{;}
\PY{n}{Singleton}\PY{+w}{ }\PY{n}{obj2}\PY{+w}{ }\PY{o}{=}\PY{+w}{ }\PY{n}{Singleton}\PY{p}{.}\PY{n}{GetInstance}\PY{p}{(}\PY{p}{)}\PY{p}{;}

\PY{c+c1}{// Student details (added at the end)}
\PY{n}{Console}\PY{p}{.}\PY{n}{WriteLine}\PY{p}{(}\PY{l+s}{\PYZdq{}\PYZbs{}nStudent Details:\PYZdq{}}\PY{p}{)}\PY{p}{;}
\PY{n}{Console}\PY{p}{.}\PY{n}{WriteLine}\PY{p}{(}\PY{l+s}{\PYZdq{}\PYZbs{}nName: Mohit Tharu\PYZdq{}}\PY{p}{)}\PY{p}{;}
\PY{n}{Console}\PY{p}{.}\PY{n}{WriteLine}\PY{p}{(}\PY{l+s}{\PYZdq{}\PYZbs{}nRoll no: 117\PYZdq{}}\PY{p}{)}\PY{p}{;}

\PY{n}{Console}\PY{p}{.}\PY{n}{ReadLine}\PY{p}{(}\PY{p}{)}\PY{p}{;}
\end{Verbatim}
\end{tcolorbox}

    \begin{Verbatim}[commandchars=\\\{\}]
Static Constructor called
Default Constructor called
Id: 0, Name: Not Assigned, College: ABC Engineering College
Parameterized Constructor called
Id: 101, Name: Rahul, College: ABC Engineering College
Copy Constructor called
Id: 101, Name: Rahul, College: ABC Engineering College
Private Constructor called

Student Details:

Name: Mohit Tharu

Roll no: 117
    \end{Verbatim}

    \hypertarget{wap-in-c-to-demonstrate-concept-of-auto-property-and-read-only-property.}{%
\section{2. WAP in C\# to demonstrate concept of Auto Property and
Read-Only
Property.}\label{wap-in-c-to-demonstrate-concept-of-auto-property-and-read-only-property.}}

Theory: Properties in C\# provide a flexible way to read, write, or
compute values of private fields. Auto Property allows automatic
creation of getter and setter without explicitly declaring a backing
variable. Read-Only Property allows data to be read but not modified
after initialization.These properties improve code readability and data
security.

    \begin{tcolorbox}[breakable, size=fbox, boxrule=1pt, pad at break*=1mm,colback=cellbackground, colframe=cellborder]
\prompt{In}{incolor}{11}{\boxspacing}
\begin{Verbatim}[commandchars=\\\{\}]
\PY{k}{using}\PY{+w}{ }\PY{n+nn}{System}\PY{p}{;}

\PY{k}{class}\PY{+w}{ }\PY{n+nc}{Student}
\PY{p}{\PYZob{}}
\PY{+w}{    }\PY{c+c1}{// Auto Properties}
\PY{+w}{    }\PY{k}{public}\PY{+w}{ }\PY{k+kt}{string}\PY{+w}{ }\PY{n}{Name}\PY{+w}{ }\PY{p}{\PYZob{}}\PY{+w}{ }\PY{k}{get}\PY{p}{;}\PY{+w}{ }\PY{k}{set}\PY{p}{;}\PY{+w}{ }\PY{p}{\PYZcb{}}
\PY{+w}{    }\PY{k}{public}\PY{+w}{ }\PY{k+kt}{int}\PY{+w}{ }\PY{n}{RollNo}\PY{+w}{ }\PY{p}{\PYZob{}}\PY{+w}{ }\PY{k}{get}\PY{p}{;}\PY{+w}{ }\PY{k}{set}\PY{p}{;}\PY{+w}{ }\PY{p}{\PYZcb{}}

\PY{+w}{    }\PY{c+c1}{// Read\PYZhy{}Only Property}
\PY{+w}{    }\PY{k}{public}\PY{+w}{ }\PY{k+kt}{string}\PY{+w}{ }\PY{n}{College}\PY{+w}{ }\PY{p}{\PYZob{}}\PY{+w}{ }\PY{k}{get}\PY{p}{;}\PY{+w}{ }\PY{p}{\PYZcb{}}

\PY{+w}{    }\PY{c+c1}{// Constructor}
\PY{+w}{    }\PY{k}{public}\PY{+w}{ }\PY{n+nf}{Student}\PY{p}{(}\PY{k+kt}{string}\PY{+w}{ }\PY{n}{name}\PY{p}{,}\PY{+w}{ }\PY{k+kt}{int}\PY{+w}{ }\PY{n}{rollNo}\PY{p}{)}
\PY{+w}{    }\PY{p}{\PYZob{}}
\PY{+w}{        }\PY{n}{Name}\PY{+w}{ }\PY{o}{=}\PY{+w}{ }\PY{n}{name}\PY{p}{;}
\PY{+w}{        }\PY{n}{RollNo}\PY{+w}{ }\PY{o}{=}\PY{+w}{ }\PY{n}{rollNo}\PY{p}{;}
\PY{+w}{        }\PY{n}{College}\PY{+w}{ }\PY{o}{=}\PY{+w}{ }\PY{l+s}{\PYZdq{}Patan Multiple Campus\PYZdq{}}\PY{p}{;}
\PY{+w}{    }\PY{p}{\PYZcb{}}

\PY{+w}{    }\PY{k}{public}\PY{+w}{ }\PY{k}{void}\PY{+w}{ }\PY{n+nf}{Display}\PY{p}{(}\PY{p}{)}
\PY{+w}{    }\PY{p}{\PYZob{}}
\PY{+w}{        }\PY{n}{Console}\PY{p}{.}\PY{n}{WriteLine}\PY{p}{(}\PY{l+s}{\PYZdq{}Student Details:\PYZdq{}}\PY{p}{)}\PY{p}{;}
\PY{+w}{        }\PY{n}{Console}\PY{p}{.}\PY{n}{WriteLine}\PY{p}{(}\PY{l+s}{\PYZdq{}Name: \PYZdq{}}\PY{+w}{ }\PY{o}{+}\PY{+w}{ }\PY{n}{Name}\PY{p}{)}\PY{p}{;}
\PY{+w}{        }\PY{n}{Console}\PY{p}{.}\PY{n}{WriteLine}\PY{p}{(}\PY{l+s}{\PYZdq{}Roll No: \PYZdq{}}\PY{+w}{ }\PY{o}{+}\PY{+w}{ }\PY{n}{RollNo}\PY{p}{)}\PY{p}{;}
\PY{+w}{        }\PY{n}{Console}\PY{p}{.}\PY{n}{WriteLine}\PY{p}{(}\PY{l+s}{\PYZdq{}College: \PYZdq{}}\PY{+w}{ }\PY{o}{+}\PY{+w}{ }\PY{n}{College}\PY{p}{)}\PY{p}{;}
\PY{+w}{    }\PY{p}{\PYZcb{}}
\PY{p}{\PYZcb{}}

\PY{c+c1}{// Top\PYZhy{}level code}
\PY{n}{Student}\PY{+w}{ }\PY{n}{s}\PY{+w}{ }\PY{o}{=}\PY{+w}{ }\PY{k}{new}\PY{+w}{ }\PY{n}{Student}\PY{p}{(}\PY{l+s}{\PYZdq{}Mohit Tharu\PYZdq{}}\PY{p}{,}\PY{+w}{ }\PY{l+m+mi}{117}\PY{p}{)}\PY{p}{;}
\PY{n}{s}\PY{p}{.}\PY{n}{Display}\PY{p}{(}\PY{p}{)}\PY{p}{;}

\PY{n}{Console}\PY{p}{.}\PY{n}{ReadLine}\PY{p}{(}\PY{p}{)}\PY{p}{;}
\end{Verbatim}
\end{tcolorbox}

    \begin{Verbatim}[commandchars=\\\{\}]
Student Details:
Name: Mohit Tharu
Roll No: 117
College: Patan Multiple Campus
    \end{Verbatim}

    \hypertarget{wap-in-c-to-demonstrate-jagged-array.}{%
\section{3. WAP in C\# to demonstrate Jagged
Array.}\label{wap-in-c-to-demonstrate-jagged-array.}}

Theory:A jagged array is an array of arrays in which each inner array
can have a different size. Unlike multidimensional arrays, jagged arrays
provide flexibility in storing uneven data and are commonly used when
rows contain varying numbers of elements.

    \begin{tcolorbox}[breakable, size=fbox, boxrule=1pt, pad at break*=1mm,colback=cellbackground, colframe=cellborder]
\prompt{In}{incolor}{19}{\boxspacing}
\begin{Verbatim}[commandchars=\\\{\}]
\PY{k}{using}\PY{+w}{ }\PY{n+nn}{System}\PY{p}{;}

\PY{c+c1}{// Declaring Jagged Array}
\PY{k+kt}{int}\PY{p}{[}\PY{p}{]}\PY{p}{[}\PY{p}{]}\PY{+w}{ }\PY{n}{marks}\PY{+w}{ }\PY{o}{=}\PY{+w}{ }\PY{k}{new}\PY{+w}{ }\PY{k+kt}{int}\PY{p}{[}\PY{l+m+mi}{3}\PY{p}{]}\PY{p}{[}\PY{p}{]}\PY{p}{;}

\PY{c+c1}{// Initializing Jagged Array}
\PY{n}{marks}\PY{p}{[}\PY{l+m+mi}{0}\PY{p}{]}\PY{+w}{ }\PY{o}{=}\PY{+w}{ }\PY{k}{new}\PY{+w}{ }\PY{k+kt}{int}\PY{p}{[}\PY{p}{]}\PY{+w}{ }\PY{p}{\PYZob{}}\PY{+w}{ }\PY{l+m+mi}{80}\PY{p}{,}\PY{+w}{ }\PY{l+m+mi}{85}\PY{p}{,}\PY{+w}{ }\PY{l+m+mi}{90}\PY{+w}{ }\PY{p}{\PYZcb{}}\PY{p}{;}
\PY{n}{marks}\PY{p}{[}\PY{l+m+mi}{1}\PY{p}{]}\PY{+w}{ }\PY{o}{=}\PY{+w}{ }\PY{k}{new}\PY{+w}{ }\PY{k+kt}{int}\PY{p}{[}\PY{p}{]}\PY{+w}{ }\PY{p}{\PYZob{}}\PY{+w}{ }\PY{l+m+mi}{75}\PY{p}{,}\PY{+w}{ }\PY{l+m+mi}{88}\PY{+w}{ }\PY{p}{\PYZcb{}}\PY{p}{;}
\PY{n}{marks}\PY{p}{[}\PY{l+m+mi}{2}\PY{p}{]}\PY{+w}{ }\PY{o}{=}\PY{+w}{ }\PY{k}{new}\PY{+w}{ }\PY{k+kt}{int}\PY{p}{[}\PY{p}{]}\PY{+w}{ }\PY{p}{\PYZob{}}\PY{+w}{ }\PY{l+m+mi}{92}\PY{p}{,}\PY{+w}{ }\PY{l+m+mi}{89}\PY{p}{,}\PY{+w}{ }\PY{l+m+mi}{95}\PY{p}{,}\PY{+w}{ }\PY{l+m+mi}{91}\PY{+w}{ }\PY{p}{\PYZcb{}}\PY{p}{;}

\PY{c+c1}{// Displaying Jagged Array elements}
\PY{n}{Console}\PY{p}{.}\PY{n}{WriteLine}\PY{p}{(}\PY{l+s}{\PYZdq{}Jagged Array Elements:\PYZdq{}}\PY{p}{)}\PY{p}{;}

\PY{k}{for}\PY{+w}{ }\PY{p}{(}\PY{k+kt}{int}\PY{+w}{ }\PY{n}{i}\PY{+w}{ }\PY{o}{=}\PY{+w}{ }\PY{l+m+mi}{0}\PY{p}{;}\PY{+w}{ }\PY{n}{i}\PY{+w}{ }\PY{o}{\PYZlt{}}\PY{+w}{ }\PY{n}{marks}\PY{p}{.}\PY{n}{Length}\PY{p}{;}\PY{+w}{ }\PY{n}{i}\PY{o}{++}\PY{p}{)}
\PY{p}{\PYZob{}}
\PY{+w}{    }\PY{n}{Console}\PY{p}{.}\PY{n}{Write}\PY{p}{(}\PY{l+s}{\PYZdq{}Row \PYZdq{}}\PY{+w}{ }\PY{o}{+}\PY{+w}{ }\PY{p}{(}\PY{n}{i}\PY{+w}{ }\PY{o}{+}\PY{+w}{ }\PY{l+m+mi}{1}\PY{p}{)}\PY{+w}{ }\PY{o}{+}\PY{+w}{ }\PY{l+s}{\PYZdq{}: \PYZdq{}}\PY{p}{)}\PY{p}{;}
\PY{+w}{    }\PY{k}{for}\PY{+w}{ }\PY{p}{(}\PY{k+kt}{int}\PY{+w}{ }\PY{n}{j}\PY{+w}{ }\PY{o}{=}\PY{+w}{ }\PY{l+m+mi}{0}\PY{p}{;}\PY{+w}{ }\PY{n}{j}\PY{+w}{ }\PY{o}{\PYZlt{}}\PY{+w}{ }\PY{n}{marks}\PY{p}{[}\PY{n}{i}\PY{p}{]}\PY{p}{.}\PY{n}{Length}\PY{p}{;}\PY{+w}{ }\PY{n}{j}\PY{o}{++}\PY{p}{)}
\PY{+w}{    }\PY{p}{\PYZob{}}
\PY{+w}{        }\PY{n}{Console}\PY{p}{.}\PY{n}{Write}\PY{p}{(}\PY{n}{marks}\PY{p}{[}\PY{n}{i}\PY{p}{]}\PY{p}{[}\PY{n}{j}\PY{p}{]}\PY{+w}{ }\PY{o}{+}\PY{+w}{ }\PY{l+s}{\PYZdq{} \PYZdq{}}\PY{p}{)}\PY{p}{;}
\PY{+w}{    }\PY{p}{\PYZcb{}}
\PY{+w}{    }\PY{n}{Console}\PY{p}{.}\PY{n}{WriteLine}\PY{p}{(}\PY{p}{)}\PY{p}{;}
\PY{p}{\PYZcb{}}

\PY{c+c1}{// Student details (added at the end)}
\PY{n}{Console}\PY{p}{.}\PY{n}{WriteLine}\PY{p}{(}\PY{l+s}{\PYZdq{}\PYZbs{}nStudent Details:\PYZdq{}}\PY{p}{)}\PY{p}{;}
\PY{n}{Console}\PY{p}{.}\PY{n}{WriteLine}\PY{p}{(}\PY{l+s}{\PYZdq{}Name: Mohit Tharu\PYZdq{}}\PY{p}{)}\PY{p}{;}
\PY{n}{Console}\PY{p}{.}\PY{n}{WriteLine}\PY{p}{(}\PY{l+s}{\PYZdq{}Roll No: 117\PYZdq{}}\PY{p}{)}\PY{p}{;}
\PY{n}{Console}\PY{p}{.}\PY{n}{WriteLine}\PY{p}{(}\PY{l+s}{\PYZdq{}College: Patan Multiple Campus\PYZdq{}}\PY{p}{)}\PY{p}{;}

\PY{n}{Console}\PY{p}{.}\PY{n}{ReadLine}\PY{p}{(}\PY{p}{)}\PY{p}{;}
\end{Verbatim}
\end{tcolorbox}

    \begin{Verbatim}[commandchars=\\\{\}]
Jagged Array Elements:
Row 1: 80 85 90
Row 2: 75 88
Row 3: 92 89 95 91

Student Details:
Name: Mohit Tharu
Roll No: 117
College: Patan Multiple Campus
    \end{Verbatim}

    \hypertarget{wap-to-demonstrate-indexer-in-ca-when-index-is-of-int-type-b-when-index-is-of-other-than-int-type}{%
\section{4. WAP to demonstrate Indexer in C\#:a) When index is of int
type b) When index is of other than int
type}\label{wap-to-demonstrate-indexer-in-ca-when-index-is-of-int-type-b-when-index-is-of-other-than-int-type}}

Theory: An Indexer allows an object of a class to be accessed like an
array. It uses the this keyword and can accept different types of
parameters. Int type indexer allows access using numeric index values.
Non-int type indexer (like string) allows access using keys or names.
Indexers improve data encapsulation and make object access easier.

    \begin{tcolorbox}[breakable, size=fbox, boxrule=1pt, pad at break*=1mm,colback=cellbackground, colframe=cellborder]
\prompt{In}{incolor}{24}{\boxspacing}
\begin{Verbatim}[commandchars=\\\{\}]
\PY{k}{using}\PY{+w}{ }\PY{n+nn}{System}\PY{p}{;}

\PY{c+c1}{// (a) Indexer with int type}
\PY{k}{class}\PY{+w}{ }\PY{n+nc}{IntIndexer}
\PY{p}{\PYZob{}}
\PY{+w}{    }\PY{k}{private}\PY{+w}{ }\PY{k+kt}{int}\PY{p}{[}\PY{p}{]}\PY{+w}{ }\PY{n}{numbers}\PY{+w}{ }\PY{o}{=}\PY{+w}{ }\PY{k}{new}\PY{+w}{ }\PY{k+kt}{int}\PY{p}{[}\PY{l+m+mi}{5}\PY{p}{]}\PY{p}{;}

\PY{+w}{    }\PY{k}{public}\PY{+w}{ }\PY{k+kt}{int}\PY{+w}{ }\PY{k}{this}\PY{p}{[}\PY{k+kt}{int}\PY{+w}{ }\PY{n}{index}\PY{p}{]}
\PY{+w}{    }\PY{p}{\PYZob{}}
\PY{+w}{        }\PY{k}{get}\PY{+w}{ }\PY{p}{\PYZob{}}\PY{+w}{ }\PY{k}{return}\PY{+w}{ }\PY{n}{numbers}\PY{p}{[}\PY{n}{index}\PY{p}{]}\PY{p}{;}\PY{+w}{ }\PY{p}{\PYZcb{}}
\PY{+w}{        }\PY{k}{set}\PY{+w}{ }\PY{p}{\PYZob{}}\PY{+w}{ }\PY{n}{numbers}\PY{p}{[}\PY{n}{index}\PY{p}{]}\PY{+w}{ }\PY{o}{=}\PY{+w}{ }\PY{k}{value}\PY{p}{;}\PY{+w}{ }\PY{p}{\PYZcb{}}
\PY{+w}{    }\PY{p}{\PYZcb{}}
\PY{p}{\PYZcb{}}

\PY{c+c1}{// (b) Indexer with non\PYZhy{}int type (string)}
\PY{k}{class}\PY{+w}{ }\PY{n+nc}{StringIndexer}
\PY{p}{\PYZob{}}
\PY{+w}{    }\PY{k}{private}\PY{+w}{ }\PY{k+kt}{string}\PY{p}{[}\PY{p}{]}\PY{+w}{ }\PY{n}{names}\PY{+w}{ }\PY{o}{=}\PY{+w}{ }\PY{k}{new}\PY{+w}{ }\PY{k+kt}{string}\PY{p}{[}\PY{l+m+mi}{3}\PY{p}{]}\PY{p}{;}

\PY{+w}{    }\PY{k}{public}\PY{+w}{ }\PY{k+kt}{string}\PY{+w}{ }\PY{k}{this}\PY{p}{[}\PY{k+kt}{string}\PY{+w}{ }\PY{n}{key}\PY{p}{]}
\PY{+w}{    }\PY{p}{\PYZob{}}
\PY{+w}{        }\PY{k}{get}
\PY{+w}{        }\PY{p}{\PYZob{}}
\PY{+w}{            }\PY{k}{if}\PY{+w}{ }\PY{p}{(}\PY{n}{key}\PY{+w}{ }\PY{o}{==}\PY{+w}{ }\PY{l+s}{\PYZdq{}first\PYZdq{}}\PY{p}{)}\PY{+w}{ }\PY{k}{return}\PY{+w}{ }\PY{n}{names}\PY{p}{[}\PY{l+m+mi}{0}\PY{p}{]}\PY{p}{;}
\PY{+w}{            }\PY{k}{else}\PY{+w}{ }\PY{n+nf}{if}\PY{+w}{ }\PY{p}{(}\PY{n}{key}\PY{+w}{ }\PY{o}{==}\PY{+w}{ }\PY{l+s}{\PYZdq{}second\PYZdq{}}\PY{p}{)}\PY{+w}{ }\PY{k}{return}\PY{+w}{ }\PY{n}{names}\PY{p}{[}\PY{l+m+mi}{1}\PY{p}{]}\PY{p}{;}
\PY{+w}{            }\PY{k}{else}\PY{+w}{ }\PY{n+nf}{if}\PY{+w}{ }\PY{p}{(}\PY{n}{key}\PY{+w}{ }\PY{o}{==}\PY{+w}{ }\PY{l+s}{\PYZdq{}third\PYZdq{}}\PY{p}{)}\PY{+w}{ }\PY{k}{return}\PY{+w}{ }\PY{n}{names}\PY{p}{[}\PY{l+m+mi}{2}\PY{p}{]}\PY{p}{;}
\PY{+w}{            }\PY{k}{else}\PY{+w}{ }\PY{k}{return}\PY{+w}{ }\PY{l+s}{\PYZdq{}Invalid Key\PYZdq{}}\PY{p}{;}
\PY{+w}{        }\PY{p}{\PYZcb{}}
\PY{+w}{        }\PY{k}{set}
\PY{+w}{        }\PY{p}{\PYZob{}}
\PY{+w}{            }\PY{k}{if}\PY{+w}{ }\PY{p}{(}\PY{n}{key}\PY{+w}{ }\PY{o}{==}\PY{+w}{ }\PY{l+s}{\PYZdq{}first\PYZdq{}}\PY{p}{)}\PY{+w}{ }\PY{n}{names}\PY{p}{[}\PY{l+m+mi}{0}\PY{p}{]}\PY{+w}{ }\PY{o}{=}\PY{+w}{ }\PY{k}{value}\PY{p}{;}
\PY{+w}{            }\PY{k}{else}\PY{+w}{ }\PY{n+nf}{if}\PY{+w}{ }\PY{p}{(}\PY{n}{key}\PY{+w}{ }\PY{o}{==}\PY{+w}{ }\PY{l+s}{\PYZdq{}second\PYZdq{}}\PY{p}{)}\PY{+w}{ }\PY{n}{names}\PY{p}{[}\PY{l+m+mi}{1}\PY{p}{]}\PY{+w}{ }\PY{o}{=}\PY{+w}{ }\PY{k}{value}\PY{p}{;}
\PY{+w}{            }\PY{k}{else}\PY{+w}{ }\PY{n+nf}{if}\PY{+w}{ }\PY{p}{(}\PY{n}{key}\PY{+w}{ }\PY{o}{==}\PY{+w}{ }\PY{l+s}{\PYZdq{}third\PYZdq{}}\PY{p}{)}\PY{+w}{ }\PY{n}{names}\PY{p}{[}\PY{l+m+mi}{2}\PY{p}{]}\PY{+w}{ }\PY{o}{=}\PY{+w}{ }\PY{k}{value}\PY{p}{;}
\PY{+w}{        }\PY{p}{\PYZcb{}}
\PY{+w}{    }\PY{p}{\PYZcb{}}
\PY{p}{\PYZcb{}}

\PY{c+c1}{// Top\PYZhy{}level code}
\PY{n}{IntIndexer}\PY{+w}{ }\PY{n}{obj1}\PY{+w}{ }\PY{o}{=}\PY{+w}{ }\PY{k}{new}\PY{+w}{ }\PY{n}{IntIndexer}\PY{p}{(}\PY{p}{)}\PY{p}{;}
\PY{n}{obj1}\PY{p}{[}\PY{l+m+mi}{0}\PY{p}{]}\PY{+w}{ }\PY{o}{=}\PY{+w}{ }\PY{l+m+mi}{10}\PY{p}{;}
\PY{n}{obj1}\PY{p}{[}\PY{l+m+mi}{1}\PY{p}{]}\PY{+w}{ }\PY{o}{=}\PY{+w}{ }\PY{l+m+mi}{20}\PY{p}{;}
\PY{n}{obj1}\PY{p}{[}\PY{l+m+mi}{2}\PY{p}{]}\PY{+w}{ }\PY{o}{=}\PY{+w}{ }\PY{l+m+mi}{30}\PY{p}{;}

\PY{n}{Console}\PY{p}{.}\PY{n}{WriteLine}\PY{p}{(}\PY{l+s}{\PYZdq{}Indexer with int type:\PYZdq{}}\PY{p}{)}\PY{p}{;}
\PY{n}{Console}\PY{p}{.}\PY{n}{WriteLine}\PY{p}{(}\PY{n}{obj1}\PY{p}{[}\PY{l+m+mi}{0}\PY{p}{]}\PY{p}{)}\PY{p}{;}
\PY{n}{Console}\PY{p}{.}\PY{n}{WriteLine}\PY{p}{(}\PY{n}{obj1}\PY{p}{[}\PY{l+m+mi}{1}\PY{p}{]}\PY{p}{)}\PY{p}{;}
\PY{n}{Console}\PY{p}{.}\PY{n}{WriteLine}\PY{p}{(}\PY{n}{obj1}\PY{p}{[}\PY{l+m+mi}{2}\PY{p}{]}\PY{p}{)}\PY{p}{;}

\PY{n}{StringIndexer}\PY{+w}{ }\PY{n}{obj2}\PY{+w}{ }\PY{o}{=}\PY{+w}{ }\PY{k}{new}\PY{+w}{ }\PY{n}{StringIndexer}\PY{p}{(}\PY{p}{)}\PY{p}{;}
\PY{n}{obj2}\PY{p}{[}\PY{l+s}{\PYZdq{}first\PYZdq{}}\PY{p}{]}\PY{+w}{ }\PY{o}{=}\PY{+w}{ }\PY{l+s}{\PYZdq{}Apple\PYZdq{}}\PY{p}{;}
\PY{n}{obj2}\PY{p}{[}\PY{l+s}{\PYZdq{}second\PYZdq{}}\PY{p}{]}\PY{+w}{ }\PY{o}{=}\PY{+w}{ }\PY{l+s}{\PYZdq{}Banana\PYZdq{}}\PY{p}{;}
\PY{n}{obj2}\PY{p}{[}\PY{l+s}{\PYZdq{}third\PYZdq{}}\PY{p}{]}\PY{+w}{ }\PY{o}{=}\PY{+w}{ }\PY{l+s}{\PYZdq{}Mango\PYZdq{}}\PY{p}{;}

\PY{n}{Console}\PY{p}{.}\PY{n}{WriteLine}\PY{p}{(}\PY{l+s}{\PYZdq{}\PYZbs{}nIndexer with non\PYZhy{}int type:\PYZdq{}}\PY{p}{)}\PY{p}{;}
\PY{n}{Console}\PY{p}{.}\PY{n}{WriteLine}\PY{p}{(}\PY{n}{obj2}\PY{p}{[}\PY{l+s}{\PYZdq{}first\PYZdq{}}\PY{p}{]}\PY{p}{)}\PY{p}{;}
\PY{n}{Console}\PY{p}{.}\PY{n}{WriteLine}\PY{p}{(}\PY{n}{obj2}\PY{p}{[}\PY{l+s}{\PYZdq{}second\PYZdq{}}\PY{p}{]}\PY{p}{)}\PY{p}{;}
\PY{n}{Console}\PY{p}{.}\PY{n}{WriteLine}\PY{p}{(}\PY{n}{obj2}\PY{p}{[}\PY{l+s}{\PYZdq{}third\PYZdq{}}\PY{p}{]}\PY{p}{)}\PY{p}{;}

\PY{c+c1}{// Student details}
\PY{n}{Console}\PY{p}{.}\PY{n}{WriteLine}\PY{p}{(}\PY{l+s}{\PYZdq{}\PYZbs{}nStudent Details:\PYZdq{}}\PY{p}{)}\PY{p}{;}
\PY{n}{Console}\PY{p}{.}\PY{n}{WriteLine}\PY{p}{(}\PY{l+s}{\PYZdq{}Name: Mohit Tharu\PYZdq{}}\PY{p}{)}\PY{p}{;}
\PY{n}{Console}\PY{p}{.}\PY{n}{WriteLine}\PY{p}{(}\PY{l+s}{\PYZdq{}Roll No: 117\PYZdq{}}\PY{p}{)}\PY{p}{;}
\PY{n}{Console}\PY{p}{.}\PY{n}{WriteLine}\PY{p}{(}\PY{l+s}{\PYZdq{}College: Patan Multiple Campus\PYZdq{}}\PY{p}{)}\PY{p}{;}

\PY{n}{Console}\PY{p}{.}\PY{n}{ReadLine}\PY{p}{(}\PY{p}{)}\PY{p}{;}
\end{Verbatim}
\end{tcolorbox}

    \begin{Verbatim}[commandchars=\\\{\}]
Indexer with int type:
10
20
30

Indexer with non-int type:
Apple
Banana
Mango

Student Details:
Name: Mohit Tharu
Roll No: 117
College: Patan Multiple Campus
    \end{Verbatim}

    \hypertarget{wap-to-demonstrate}{%
\section{5. WAP to demonstrate:}\label{wap-to-demonstrate}}

\begin{verbatim}
# a) The use of base keyword to access base class fields 
# b) The use of base keyword to call base class methods
# c) The use of base keyword to call base class constructor
\end{verbatim}

Theory: The base keyword in C\# is used to access members of the base
(parent) class from a derived (child) class. Uses of base keyword:
Access base class fields when they are hidden by derived class members.
Call base class methods from the derived class.

    \begin{tcolorbox}[breakable, size=fbox, boxrule=1pt, pad at break*=1mm,colback=cellbackground, colframe=cellborder]
\prompt{In}{incolor}{30}{\boxspacing}
\begin{Verbatim}[commandchars=\\\{\}]
\PY{k}{using}\PY{+w}{ }\PY{n+nn}{System}\PY{p}{;}

\PY{c+c1}{// Base class}
\PY{k}{class}\PY{+w}{ }\PY{n+nc}{Person}
\PY{p}{\PYZob{}}
\PY{+w}{    }\PY{k}{protected}\PY{+w}{ }\PY{k+kt}{string}\PY{+w}{ }\PY{n}{name}\PY{p}{;}

\PY{+w}{    }\PY{c+c1}{// Base class constructor}
\PY{+w}{    }\PY{k}{public}\PY{+w}{ }\PY{n+nf}{Person}\PY{p}{(}\PY{k+kt}{string}\PY{+w}{ }\PY{n}{name}\PY{p}{)}
\PY{+w}{    }\PY{p}{\PYZob{}}
\PY{+w}{        }\PY{k}{this}\PY{p}{.}\PY{n}{name}\PY{+w}{ }\PY{o}{=}\PY{+w}{ }\PY{n}{name}\PY{p}{;}
\PY{+w}{        }\PY{n}{Console}\PY{p}{.}\PY{n}{WriteLine}\PY{p}{(}\PY{l+s}{\PYZdq{}Base class constructor called\PYZdq{}}\PY{p}{)}\PY{p}{;}
\PY{+w}{    }\PY{p}{\PYZcb{}}

\PY{+w}{    }\PY{c+c1}{// Base class method}
\PY{+w}{    }\PY{k}{public}\PY{+w}{ }\PY{k}{void}\PY{+w}{ }\PY{n+nf}{ShowName}\PY{p}{(}\PY{p}{)}
\PY{+w}{    }\PY{p}{\PYZob{}}
\PY{+w}{        }\PY{n}{Console}\PY{p}{.}\PY{n}{WriteLine}\PY{p}{(}\PY{l+s}{\PYZdq{}Name from base class: \PYZdq{}}\PY{+w}{ }\PY{o}{+}\PY{+w}{ }\PY{n}{name}\PY{p}{)}\PY{p}{;}
\PY{+w}{    }\PY{p}{\PYZcb{}}
\PY{p}{\PYZcb{}}

\PY{c+c1}{// Derived class}
\PY{k}{class}\PY{+w}{ }\PY{n+nc}{Student}\PY{+w}{ }\PY{p}{:}\PY{+w}{ }\PY{n}{Person}
\PY{p}{\PYZob{}}
\PY{+w}{    }\PY{k}{public}\PY{+w}{ }\PY{k+kt}{int}\PY{+w}{ }\PY{n}{RollNo}\PY{p}{;}

\PY{+w}{    }\PY{c+c1}{// Derived class constructor calling base class constructor}
\PY{+w}{    }\PY{k}{public}\PY{+w}{ }\PY{n+nf}{Student}\PY{p}{(}\PY{k+kt}{string}\PY{+w}{ }\PY{n}{name}\PY{p}{,}\PY{+w}{ }\PY{k+kt}{int}\PY{+w}{ }\PY{n}{rollNo}\PY{p}{)}\PY{+w}{ }\PY{p}{:}\PY{+w}{ }\PY{k}{base}\PY{p}{(}\PY{n}{name}\PY{p}{)}
\PY{+w}{    }\PY{p}{\PYZob{}}
\PY{+w}{        }\PY{n}{RollNo}\PY{+w}{ }\PY{o}{=}\PY{+w}{ }\PY{n}{rollNo}\PY{p}{;}
\PY{+w}{        }\PY{n}{Console}\PY{p}{.}\PY{n}{WriteLine}\PY{p}{(}\PY{l+s}{\PYZdq{}Derived class constructor called\PYZdq{}}\PY{p}{)}\PY{p}{;}
\PY{+w}{    }\PY{p}{\PYZcb{}}

\PY{+w}{    }\PY{k}{public}\PY{+w}{ }\PY{k}{void}\PY{+w}{ }\PY{n+nf}{Display}\PY{p}{(}\PY{p}{)}
\PY{+w}{    }\PY{p}{\PYZob{}}
\PY{+w}{        }\PY{c+c1}{// (a) Access base class field using base keyword}
\PY{+w}{        }\PY{n}{Console}\PY{p}{.}\PY{n}{WriteLine}\PY{p}{(}\PY{l+s}{\PYZdq{}Accessing base class field using base keyword: \PYZdq{}}\PY{+w}{ }\PY{o}{+}\PY{+w}{ }\PY{k}{base}\PY{p}{.}\PY{n}{name}\PY{p}{)}\PY{p}{;}

\PY{+w}{        }\PY{c+c1}{// (b) Call base class method using base keyword}
\PY{+w}{        }\PY{k}{base}\PY{p}{.}\PY{n}{ShowName}\PY{p}{(}\PY{p}{)}\PY{p}{;}

\PY{+w}{        }\PY{n}{Console}\PY{p}{.}\PY{n}{WriteLine}\PY{p}{(}\PY{l+s}{\PYZdq{}Roll No: \PYZdq{}}\PY{+w}{ }\PY{o}{+}\PY{+w}{ }\PY{n}{RollNo}\PY{p}{)}\PY{p}{;}
\PY{+w}{    }\PY{p}{\PYZcb{}}
\PY{p}{\PYZcb{}}

\PY{c+c1}{// Top\PYZhy{}level code}
\PY{n}{Student}\PY{+w}{ }\PY{n}{s}\PY{+w}{ }\PY{o}{=}\PY{+w}{ }\PY{k}{new}\PY{+w}{ }\PY{n}{Student}\PY{p}{(}\PY{l+s}{\PYZdq{}Mohit Tharu\PYZdq{}}\PY{p}{,}\PY{+w}{ }\PY{l+m+mi}{117}\PY{p}{)}\PY{p}{;}
\PY{n}{s}\PY{p}{.}\PY{n}{Display}\PY{p}{(}\PY{p}{)}\PY{p}{;}

\PY{c+c1}{// Student details}
\PY{n}{Console}\PY{p}{.}\PY{n}{WriteLine}\PY{p}{(}\PY{l+s}{\PYZdq{}\PYZbs{}nStudent Details:\PYZdq{}}\PY{p}{)}\PY{p}{;}
\PY{n}{Console}\PY{p}{.}\PY{n}{WriteLine}\PY{p}{(}\PY{l+s}{\PYZdq{}Name: Mohit Tharu\PYZdq{}}\PY{p}{)}\PY{p}{;}
\PY{n}{Console}\PY{p}{.}\PY{n}{WriteLine}\PY{p}{(}\PY{l+s}{\PYZdq{}Roll No: 117\PYZdq{}}\PY{p}{)}\PY{p}{;}
\PY{n}{Console}\PY{p}{.}\PY{n}{WriteLine}\PY{p}{(}\PY{l+s}{\PYZdq{}College: Patan Multiple Campus\PYZdq{}}\PY{p}{)}\PY{p}{;}

\PY{n}{Console}\PY{p}{.}\PY{n}{ReadLine}\PY{p}{(}\PY{p}{)}\PY{p}{;}
\end{Verbatim}
\end{tcolorbox}

    \begin{Verbatim}[commandchars=\\\{\}]
Base class constructor called
Derived class constructor called
Accessing base class field using base keyword: Mohit Tharu
Name from base class: Mohit Tharu
Roll No: 117

Student Details:
Name: Mohit Tharu
Roll No: 117
College: Patan Multiple Campus
    \end{Verbatim}

    \hypertarget{program-to-show-a-method-overriding-and-method-hidingshadowing-in-c-b-dynamic-polymorphism-using-method-overriding.}{%
\section{6. Program to show a) method overriding and method
hiding/shadowing in C\# b) dynamic polymorphism using method
overriding.}\label{program-to-show-a-method-overriding-and-method-hidingshadowing-in-c-b-dynamic-polymorphism-using-method-overriding.}}

Theory: Method Overriding occurs when a derived class provides a new
implementation of a base class method using virtual and override
keywords. Method Hiding (Shadowing) occurs when a derived class defines
a method with the same name as the base class method using the new
keyword. Dynamic Polymorphism means that the method call is resolved at
runtime based on the object type, achieved using method overriding.

    \begin{tcolorbox}[breakable, size=fbox, boxrule=1pt, pad at break*=1mm,colback=cellbackground, colframe=cellborder]
\prompt{In}{incolor}{33}{\boxspacing}
\begin{Verbatim}[commandchars=\\\{\}]
\PY{k}{using}\PY{+w}{ }\PY{n+nn}{System}\PY{p}{;}

\PY{c+c1}{// Base class}
\PY{k}{class}\PY{+w}{ }\PY{n+nc}{Animal}
\PY{p}{\PYZob{}}
\PY{+w}{    }\PY{k}{public}\PY{+w}{ }\PY{k}{virtual}\PY{+w}{ }\PY{k}{void}\PY{+w}{ }\PY{n+nf}{Sound}\PY{p}{(}\PY{p}{)}
\PY{+w}{    }\PY{p}{\PYZob{}}
\PY{+w}{        }\PY{n}{Console}\PY{p}{.}\PY{n}{WriteLine}\PY{p}{(}\PY{l+s}{\PYZdq{}Animal makes a sound\PYZdq{}}\PY{p}{)}\PY{p}{;}
\PY{+w}{    }\PY{p}{\PYZcb{}}

\PY{+w}{    }\PY{k}{public}\PY{+w}{ }\PY{k}{void}\PY{+w}{ }\PY{n+nf}{Eat}\PY{p}{(}\PY{p}{)}
\PY{+w}{    }\PY{p}{\PYZob{}}
\PY{+w}{        }\PY{n}{Console}\PY{p}{.}\PY{n}{WriteLine}\PY{p}{(}\PY{l+s}{\PYZdq{}Animal eats food\PYZdq{}}\PY{p}{)}\PY{p}{;}
\PY{+w}{    }\PY{p}{\PYZcb{}}
\PY{p}{\PYZcb{}}

\PY{c+c1}{// Derived class}
\PY{k}{class}\PY{+w}{ }\PY{n+nc}{Dog}\PY{+w}{ }\PY{p}{:}\PY{+w}{ }\PY{n}{Animal}
\PY{p}{\PYZob{}}
\PY{+w}{    }\PY{c+c1}{// Method Overriding}
\PY{+w}{    }\PY{k}{public}\PY{+w}{ }\PY{k}{override}\PY{+w}{ }\PY{k}{void}\PY{+w}{ }\PY{n+nf}{Sound}\PY{p}{(}\PY{p}{)}
\PY{+w}{    }\PY{p}{\PYZob{}}
\PY{+w}{        }\PY{n}{Console}\PY{p}{.}\PY{n}{WriteLine}\PY{p}{(}\PY{l+s}{\PYZdq{}Dog barks\PYZdq{}}\PY{p}{)}\PY{p}{;}
\PY{+w}{    }\PY{p}{\PYZcb{}}

\PY{+w}{    }\PY{c+c1}{// Method Hiding / Shadowing}
\PY{+w}{    }\PY{k}{public}\PY{+w}{ }\PY{k}{new}\PY{+w}{ }\PY{k}{void}\PY{+w}{ }\PY{n+nf}{Eat}\PY{p}{(}\PY{p}{)}
\PY{+w}{    }\PY{p}{\PYZob{}}
\PY{+w}{        }\PY{n}{Console}\PY{p}{.}\PY{n}{WriteLine}\PY{p}{(}\PY{l+s}{\PYZdq{}Dog eats bones\PYZdq{}}\PY{p}{)}\PY{p}{;}
\PY{+w}{    }\PY{p}{\PYZcb{}}
\PY{p}{\PYZcb{}}

\PY{c+c1}{// Top\PYZhy{}level code}
\PY{n}{Animal}\PY{+w}{ }\PY{n}{a1}\PY{+w}{ }\PY{o}{=}\PY{+w}{ }\PY{k}{new}\PY{+w}{ }\PY{n}{Animal}\PY{p}{(}\PY{p}{)}\PY{p}{;}
\PY{n}{Animal}\PY{+w}{ }\PY{n}{a2}\PY{+w}{ }\PY{o}{=}\PY{+w}{ }\PY{k}{new}\PY{+w}{ }\PY{n}{Dog}\PY{p}{(}\PY{p}{)}\PY{p}{;}\PY{+w}{   }\PY{c+c1}{// Dynamic polymorphism}
\PY{n}{Dog}\PY{+w}{ }\PY{n}{d}\PY{+w}{ }\PY{o}{=}\PY{+w}{ }\PY{k}{new}\PY{+w}{ }\PY{n}{Dog}\PY{p}{(}\PY{p}{)}\PY{p}{;}

\PY{n}{Console}\PY{p}{.}\PY{n}{WriteLine}\PY{p}{(}\PY{l+s}{\PYZdq{}Method Overriding \PYZam{} Dynamic Polymorphism:\PYZdq{}}\PY{p}{)}\PY{p}{;}
\PY{n}{a1}\PY{p}{.}\PY{n}{Sound}\PY{p}{(}\PY{p}{)}\PY{p}{;}\PY{+w}{   }\PY{c+c1}{// Base class method}
\PY{n}{a2}\PY{p}{.}\PY{n}{Sound}\PY{p}{(}\PY{p}{)}\PY{p}{;}\PY{+w}{   }\PY{c+c1}{// Derived class method (runtime binding)}

\PY{n}{Console}\PY{p}{.}\PY{n}{WriteLine}\PY{p}{(}\PY{l+s}{\PYZdq{}\PYZbs{}nMethod Hiding / Shadowing:\PYZdq{}}\PY{p}{)}\PY{p}{;}
\PY{n}{a2}\PY{p}{.}\PY{n}{Eat}\PY{p}{(}\PY{p}{)}\PY{p}{;}\PY{+w}{     }\PY{c+c1}{// Base class method}
\PY{n}{d}\PY{p}{.}\PY{n}{Eat}\PY{p}{(}\PY{p}{)}\PY{p}{;}\PY{+w}{      }\PY{c+c1}{// Derived class method}

\PY{c+c1}{// Student details}
\PY{n}{Console}\PY{p}{.}\PY{n}{WriteLine}\PY{p}{(}\PY{l+s}{\PYZdq{}\PYZbs{}nStudent Details:\PYZdq{}}\PY{p}{)}\PY{p}{;}
\PY{n}{Console}\PY{p}{.}\PY{n}{WriteLine}\PY{p}{(}\PY{l+s}{\PYZdq{}Name: Mohit Tharu\PYZdq{}}\PY{p}{)}\PY{p}{;}
\PY{n}{Console}\PY{p}{.}\PY{n}{WriteLine}\PY{p}{(}\PY{l+s}{\PYZdq{}Roll No: 117\PYZdq{}}\PY{p}{)}\PY{p}{;}
\PY{n}{Console}\PY{p}{.}\PY{n}{WriteLine}\PY{p}{(}\PY{l+s}{\PYZdq{}College: Patan Multiple Campus\PYZdq{}}\PY{p}{)}\PY{p}{;}

\PY{n}{Console}\PY{p}{.}\PY{n}{ReadLine}\PY{p}{(}\PY{p}{)}\PY{p}{;}
\end{Verbatim}
\end{tcolorbox}

    \begin{Verbatim}[commandchars=\\\{\}]
Method Overriding \& Dynamic Polymorphism:
Animal makes a sound
Dog barks

Method Hiding / Shadowing:
Animal eats food
Dog eats bones

Student Details:
Name: Mohit Tharu
Roll No: 117
College: Patan Multiple Campus
    \end{Verbatim}

    \hypertarget{wap-to-illustrate-the-concept-of-a-abstract-class-b-interface-c-multiple-inheritance-using-interface-in-c}{%
\section{7. WAP to illustrate the concept of a) Abstract class b)
Interface c) Multiple inheritance using interface in
C}\label{wap-to-illustrate-the-concept-of-a-abstract-class-b-interface-c-multiple-inheritance-using-interface-in-c}}

Theory: An abstract class contains abstract methods (without body) and
non-abstract methods. It cannot be instantiated and must be inherited.
An interface contains only method declarations. It supports full
abstraction. Multiple inheritance is not supported using classes in C\#,
but it is achieved using multiple interfaces.

    \begin{tcolorbox}[breakable, size=fbox, boxrule=1pt, pad at break*=1mm,colback=cellbackground, colframe=cellborder]
\prompt{In}{incolor}{36}{\boxspacing}
\begin{Verbatim}[commandchars=\\\{\}]
\PY{k}{using}\PY{+w}{ }\PY{n+nn}{System}\PY{p}{;}

\PY{c+c1}{// (a) Abstract class}
\PY{k}{abstract}\PY{+w}{ }\PY{k}{class}\PY{+w}{ }\PY{n+nc}{Shape}
\PY{p}{\PYZob{}}
\PY{+w}{    }\PY{k}{public}\PY{+w}{ }\PY{k}{abstract}\PY{+w}{ }\PY{k}{void}\PY{+w}{ }\PY{n+nf}{Draw}\PY{p}{(}\PY{p}{)}\PY{p}{;}

\PY{+w}{    }\PY{k}{public}\PY{+w}{ }\PY{k}{void}\PY{+w}{ }\PY{n+nf}{Info}\PY{p}{(}\PY{p}{)}
\PY{+w}{    }\PY{p}{\PYZob{}}
\PY{+w}{        }\PY{n}{Console}\PY{p}{.}\PY{n}{WriteLine}\PY{p}{(}\PY{l+s}{\PYZdq{}This is a shape\PYZdq{}}\PY{p}{)}\PY{p}{;}
\PY{+w}{    }\PY{p}{\PYZcb{}}
\PY{p}{\PYZcb{}}

\PY{c+c1}{// (b) Interface}
\PY{k}{interface}\PY{+w}{ }\PY{n}{IPrintable}
\PY{p}{\PYZob{}}
\PY{+w}{    }\PY{k}{void}\PY{+w}{ }\PY{n+nf}{Print}\PY{p}{(}\PY{p}{)}\PY{p}{;}
\PY{p}{\PYZcb{}}

\PY{c+c1}{// Another interface (for multiple inheritance)}
\PY{k}{interface}\PY{+w}{ }\PY{n}{IColor}
\PY{p}{\PYZob{}}
\PY{+w}{    }\PY{k}{void}\PY{+w}{ }\PY{n+nf}{Color}\PY{p}{(}\PY{p}{)}\PY{p}{;}
\PY{p}{\PYZcb{}}

\PY{c+c1}{// (c) Multiple inheritance using interface}
\PY{k}{class}\PY{+w}{ }\PY{n+nc}{Circle}\PY{+w}{ }\PY{p}{:}\PY{+w}{ }\PY{n}{Shape}\PY{p}{,}\PY{+w}{ }\PY{n}{IPrintable}\PY{p}{,}\PY{+w}{ }\PY{n}{IColor}
\PY{p}{\PYZob{}}
\PY{+w}{    }\PY{k}{public}\PY{+w}{ }\PY{k}{override}\PY{+w}{ }\PY{k}{void}\PY{+w}{ }\PY{n+nf}{Draw}\PY{p}{(}\PY{p}{)}
\PY{+w}{    }\PY{p}{\PYZob{}}
\PY{+w}{        }\PY{n}{Console}\PY{p}{.}\PY{n}{WriteLine}\PY{p}{(}\PY{l+s}{\PYZdq{}Drawing Circle\PYZdq{}}\PY{p}{)}\PY{p}{;}
\PY{+w}{    }\PY{p}{\PYZcb{}}

\PY{+w}{    }\PY{k}{public}\PY{+w}{ }\PY{k}{void}\PY{+w}{ }\PY{n+nf}{Print}\PY{p}{(}\PY{p}{)}
\PY{+w}{    }\PY{p}{\PYZob{}}
\PY{+w}{        }\PY{n}{Console}\PY{p}{.}\PY{n}{WriteLine}\PY{p}{(}\PY{l+s}{\PYZdq{}Printing Circle\PYZdq{}}\PY{p}{)}\PY{p}{;}
\PY{+w}{    }\PY{p}{\PYZcb{}}

\PY{+w}{    }\PY{k}{public}\PY{+w}{ }\PY{k}{void}\PY{+w}{ }\PY{n+nf}{Color}\PY{p}{(}\PY{p}{)}
\PY{+w}{    }\PY{p}{\PYZob{}}
\PY{+w}{        }\PY{n}{Console}\PY{p}{.}\PY{n}{WriteLine}\PY{p}{(}\PY{l+s}{\PYZdq{}Circle color is Red\PYZdq{}}\PY{p}{)}\PY{p}{;}
\PY{+w}{    }\PY{p}{\PYZcb{}}
\PY{p}{\PYZcb{}}

\PY{c+c1}{// Top\PYZhy{}level code}
\PY{n}{Circle}\PY{+w}{ }\PY{n}{c}\PY{+w}{ }\PY{o}{=}\PY{+w}{ }\PY{k}{new}\PY{+w}{ }\PY{n}{Circle}\PY{p}{(}\PY{p}{)}\PY{p}{;}
\PY{n}{c}\PY{p}{.}\PY{n}{Info}\PY{p}{(}\PY{p}{)}\PY{p}{;}\PY{+w}{     }\PY{c+c1}{// Abstract class method}
\PY{n}{c}\PY{p}{.}\PY{n}{Draw}\PY{p}{(}\PY{p}{)}\PY{p}{;}\PY{+w}{     }\PY{c+c1}{// Abstract method implementation}
\PY{n}{c}\PY{p}{.}\PY{n}{Print}\PY{p}{(}\PY{p}{)}\PY{p}{;}\PY{+w}{    }\PY{c+c1}{// Interface method}
\PY{n}{c}\PY{p}{.}\PY{n}{Color}\PY{p}{(}\PY{p}{)}\PY{p}{;}\PY{+w}{    }\PY{c+c1}{// Multiple inheritance using interface}

\PY{c+c1}{// Student details}
\PY{n}{Console}\PY{p}{.}\PY{n}{WriteLine}\PY{p}{(}\PY{l+s}{\PYZdq{}\PYZbs{}nStudent Details:\PYZdq{}}\PY{p}{)}\PY{p}{;}
\PY{n}{Console}\PY{p}{.}\PY{n}{WriteLine}\PY{p}{(}\PY{l+s}{\PYZdq{}Name: Mohit Tharu\PYZdq{}}\PY{p}{)}\PY{p}{;}
\PY{n}{Console}\PY{p}{.}\PY{n}{WriteLine}\PY{p}{(}\PY{l+s}{\PYZdq{}Roll No: 117\PYZdq{}}\PY{p}{)}\PY{p}{;}
\PY{n}{Console}\PY{p}{.}\PY{n}{WriteLine}\PY{p}{(}\PY{l+s}{\PYZdq{}College: Patan Multiple Campus\PYZdq{}}\PY{p}{)}\PY{p}{;}

\PY{n}{Console}\PY{p}{.}\PY{n}{ReadLine}\PY{p}{(}\PY{p}{)}\PY{p}{;}
\end{Verbatim}
\end{tcolorbox}

    \begin{Verbatim}[commandchars=\\\{\}]
This is a shape
Drawing Circle
Printing Circle
Circle color is Red

Student Details:
Name: Mohit Tharu
Roll No: 117
College: Patan Multiple Campus
    \end{Verbatim}

    \hypertarget{wap-program-that-contains-a-structure-struct-b-enumeration-enum-c-partial-class}{%
\section{8. WAP program that contains: a) Structure (struct) b)
Enumeration (enum) c) Partial
class}\label{wap-program-that-contains-a-structure-struct-b-enumeration-enum-c-partial-class}}

Theory: A Structure (struct) is a value type used to group related
variables of different data types. An Enumeration (enum) is a
user-defined data type that consists of named constants. A Partial class
allows a class to be divided into multiple parts, which can be defined
in the same or different files.

    \begin{tcolorbox}[breakable, size=fbox, boxrule=1pt, pad at break*=1mm,colback=cellbackground, colframe=cellborder]
\prompt{In}{incolor}{39}{\boxspacing}
\begin{Verbatim}[commandchars=\\\{\}]
\PY{k}{using}\PY{+w}{ }\PY{n+nn}{System}\PY{p}{;}

\PY{c+c1}{// (a) Structure}
\PY{k}{struct}\PY{+w}{ }\PY{n+nc}{StudentStruct}
\PY{p}{\PYZob{}}
\PY{+w}{    }\PY{k}{public}\PY{+w}{ }\PY{k+kt}{string}\PY{+w}{ }\PY{n}{Name}\PY{p}{;}
\PY{+w}{    }\PY{k}{public}\PY{+w}{ }\PY{k+kt}{int}\PY{+w}{ }\PY{n}{RollNo}\PY{p}{;}

\PY{+w}{    }\PY{k}{public}\PY{+w}{ }\PY{k}{void}\PY{+w}{ }\PY{n+nf}{Display}\PY{p}{(}\PY{p}{)}
\PY{+w}{    }\PY{p}{\PYZob{}}
\PY{+w}{        }\PY{n}{Console}\PY{p}{.}\PY{n}{WriteLine}\PY{p}{(}\PY{l+s}{\PYZdq{}Struct Student:\PYZdq{}}\PY{p}{)}\PY{p}{;}
\PY{+w}{        }\PY{n}{Console}\PY{p}{.}\PY{n}{WriteLine}\PY{p}{(}\PY{l+s}{\PYZdq{}Name: \PYZdq{}}\PY{+w}{ }\PY{o}{+}\PY{+w}{ }\PY{n}{Name}\PY{p}{)}\PY{p}{;}
\PY{+w}{        }\PY{n}{Console}\PY{p}{.}\PY{n}{WriteLine}\PY{p}{(}\PY{l+s}{\PYZdq{}Roll No: \PYZdq{}}\PY{+w}{ }\PY{o}{+}\PY{+w}{ }\PY{n}{RollNo}\PY{p}{)}\PY{p}{;}
\PY{+w}{    }\PY{p}{\PYZcb{}}
\PY{p}{\PYZcb{}}

\PY{c+c1}{// (b) Enumeration}
\PY{k}{enum}\PY{+w}{ }\PY{n}{Department}
\PY{p}{\PYZob{}}
\PY{+w}{    }\PY{n}{ComputerScience}\PY{p}{,}
\PY{+w}{    }\PY{n}{InformationTechnology}\PY{p}{,}
\PY{+w}{    }\PY{n}{Electronics}
\PY{p}{\PYZcb{}}

\PY{c+c1}{// (c) Partial class (Part 1)}
\PY{k}{partial}\PY{+w}{ }\PY{k}{class}\PY{+w}{ }\PY{n+nc}{Student}
\PY{p}{\PYZob{}}
\PY{+w}{    }\PY{k}{public}\PY{+w}{ }\PY{k+kt}{string}\PY{+w}{ }\PY{n}{Name}\PY{p}{;}
\PY{+w}{    }\PY{k}{public}\PY{+w}{ }\PY{k+kt}{int}\PY{+w}{ }\PY{n}{RollNo}\PY{p}{;}
\PY{p}{\PYZcb{}}

\PY{c+c1}{// Partial class (Part 2)}
\PY{k}{partial}\PY{+w}{ }\PY{k}{class}\PY{+w}{ }\PY{n+nc}{Student}
\PY{p}{\PYZob{}}
\PY{+w}{    }\PY{k}{public}\PY{+w}{ }\PY{k}{void}\PY{+w}{ }\PY{n+nf}{Show}\PY{p}{(}\PY{p}{)}
\PY{+w}{    }\PY{p}{\PYZob{}}
\PY{+w}{        }\PY{n}{Console}\PY{p}{.}\PY{n}{WriteLine}\PY{p}{(}\PY{l+s}{\PYZdq{}\PYZbs{}nPartial Class Student:\PYZdq{}}\PY{p}{)}\PY{p}{;}
\PY{+w}{        }\PY{n}{Console}\PY{p}{.}\PY{n}{WriteLine}\PY{p}{(}\PY{l+s}{\PYZdq{}Name: \PYZdq{}}\PY{+w}{ }\PY{o}{+}\PY{+w}{ }\PY{n}{Name}\PY{p}{)}\PY{p}{;}
\PY{+w}{        }\PY{n}{Console}\PY{p}{.}\PY{n}{WriteLine}\PY{p}{(}\PY{l+s}{\PYZdq{}Roll No: \PYZdq{}}\PY{+w}{ }\PY{o}{+}\PY{+w}{ }\PY{n}{RollNo}\PY{p}{)}\PY{p}{;}
\PY{+w}{    }\PY{p}{\PYZcb{}}
\PY{p}{\PYZcb{}}

\PY{c+c1}{// Top\PYZhy{}level code}

\PY{c+c1}{// Using Structure}
\PY{n}{StudentStruct}\PY{+w}{ }\PY{n}{ss}\PY{p}{;}
\PY{n}{ss}\PY{p}{.}\PY{n}{Name}\PY{+w}{ }\PY{o}{=}\PY{+w}{ }\PY{l+s}{\PYZdq{}Mohit Tharu\PYZdq{}}\PY{p}{;}
\PY{n}{ss}\PY{p}{.}\PY{n}{RollNo}\PY{+w}{ }\PY{o}{=}\PY{+w}{ }\PY{l+m+mi}{117}\PY{p}{;}
\PY{n}{ss}\PY{p}{.}\PY{n}{Display}\PY{p}{(}\PY{p}{)}\PY{p}{;}

\PY{c+c1}{// Using Enumeration}
\PY{n}{Department}\PY{+w}{ }\PY{n}{dept}\PY{+w}{ }\PY{o}{=}\PY{+w}{ }\PY{n}{Department}\PY{p}{.}\PY{n}{ComputerScience}\PY{p}{;}
\PY{n}{Console}\PY{p}{.}\PY{n}{WriteLine}\PY{p}{(}\PY{l+s}{\PYZdq{}\PYZbs{}nEnum Department:\PYZdq{}}\PY{p}{)}\PY{p}{;}
\PY{n}{Console}\PY{p}{.}\PY{n}{WriteLine}\PY{p}{(}\PY{l+s}{\PYZdq{}Department: \PYZdq{}}\PY{+w}{ }\PY{o}{+}\PY{+w}{ }\PY{n}{dept}\PY{p}{)}\PY{p}{;}

\PY{c+c1}{// Using Partial Class}
\PY{n}{Student}\PY{+w}{ }\PY{n}{s}\PY{+w}{ }\PY{o}{=}\PY{+w}{ }\PY{k}{new}\PY{+w}{ }\PY{n}{Student}\PY{p}{(}\PY{p}{)}\PY{p}{;}
\PY{n}{s}\PY{p}{.}\PY{n}{Name}\PY{+w}{ }\PY{o}{=}\PY{+w}{ }\PY{l+s}{\PYZdq{}Mohit Tharu\PYZdq{}}\PY{p}{;}
\PY{n}{s}\PY{p}{.}\PY{n}{RollNo}\PY{+w}{ }\PY{o}{=}\PY{+w}{ }\PY{l+m+mi}{117}\PY{p}{;}
\PY{n}{s}\PY{p}{.}\PY{n}{Show}\PY{p}{(}\PY{p}{)}\PY{p}{;}

\PY{c+c1}{// Student details}
\PY{n}{Console}\PY{p}{.}\PY{n}{WriteLine}\PY{p}{(}\PY{l+s}{\PYZdq{}\PYZbs{}nStudent Details:\PYZdq{}}\PY{p}{)}\PY{p}{;}
\PY{n}{Console}\PY{p}{.}\PY{n}{WriteLine}\PY{p}{(}\PY{l+s}{\PYZdq{}Name: Mohit Tharu\PYZdq{}}\PY{p}{)}\PY{p}{;}
\PY{n}{Console}\PY{p}{.}\PY{n}{WriteLine}\PY{p}{(}\PY{l+s}{\PYZdq{}Roll No: 117\PYZdq{}}\PY{p}{)}\PY{p}{;}
\PY{n}{Console}\PY{p}{.}\PY{n}{WriteLine}\PY{p}{(}\PY{l+s}{\PYZdq{}College: Patan Multiple Campus\PYZdq{}}\PY{p}{)}\PY{p}{;}

\PY{n}{Console}\PY{p}{.}\PY{n}{ReadLine}\PY{p}{(}\PY{p}{)}\PY{p}{;}
\end{Verbatim}
\end{tcolorbox}

    \begin{Verbatim}[commandchars=\\\{\}]
Struct Student:
Name: Mohit Tharu
Roll No: 117

Enum Department:
Department: ComputerScience

Partial Class Student:
Name: Mohit Tharu
Roll No: 117

Student Details:
Name: Mohit Tharu
Roll No: 117
College: Patan Multiple Campus
    \end{Verbatim}

    \hypertarget{wap-to-illustrate-the-concept-of-a-delegate-b-multicast-delegate-c-func-delegate-d-action-delegate-e-anonymous-method-f-event-in-c.}{%
\section{9. WAP to illustrate the concept of: a) Delegate b) Multicast
delegate c) Func Delegate d) Action Delegate e) Anonymous Method f)
Event in
C\#.}\label{wap-to-illustrate-the-concept-of-a-delegate-b-multicast-delegate-c-func-delegate-d-action-delegate-e-anonymous-method-f-event-in-c.}}

Theory: Delegate: A type that references a method with a specific
signature. It allows methods to be passed as parameters. Multicast
Delegate: A delegate that references multiple methods and calls them in
sequence. Func Delegate: A generic delegate that returns a value. Action
Delegate: A generic delegate that does not return a value. Anonymous
Method: A method without a name, defined inline using delegate keyword.
Event: A mechanism that allows a class to notify other classes when
something happens, typically using delegates.

    \begin{tcolorbox}[breakable, size=fbox, boxrule=1pt, pad at break*=1mm,colback=cellbackground, colframe=cellborder]
\prompt{In}{incolor}{44}{\boxspacing}
\begin{Verbatim}[commandchars=\\\{\}]
\PY{k}{using}\PY{+w}{ }\PY{n+nn}{System}\PY{p}{;}

\PY{c+c1}{// (a) Delegate}
\PY{k}{delegate}\PY{+w}{ }\PY{k}{void}\PY{+w}{ }\PY{n+nf}{MyDelegate}\PY{p}{(}\PY{k+kt}{string}\PY{+w}{ }\PY{n}{message}\PY{p}{)}\PY{p}{;}

\PY{c+c1}{// Publisher class for Event}
\PY{k}{class}\PY{+w}{ }\PY{n+nc}{EventDemo}
\PY{p}{\PYZob{}}
\PY{+w}{    }\PY{c+c1}{// (f) Event}
\PY{+w}{    }\PY{k}{public}\PY{+w}{ }\PY{k}{event}\PY{+w}{ }\PY{n}{MyDelegate}\PY{+w}{ }\PY{n}{MyEvent}\PY{p}{;}

\PY{+w}{    }\PY{k}{public}\PY{+w}{ }\PY{k}{void}\PY{+w}{ }\PY{n+nf}{RaiseEvent}\PY{p}{(}\PY{p}{)}
\PY{+w}{    }\PY{p}{\PYZob{}}
\PY{+w}{        }\PY{k}{if}\PY{+w}{ }\PY{p}{(}\PY{n}{MyEvent}\PY{+w}{ }\PY{o}{!=}\PY{+w}{ }\PY{k}{null}\PY{p}{)}
\PY{+w}{            }\PY{n}{MyEvent}\PY{p}{(}\PY{l+s}{\PYZdq{}Event has been triggered\PYZdq{}}\PY{p}{)}\PY{p}{;}
\PY{+w}{    }\PY{p}{\PYZcb{}}
\PY{p}{\PYZcb{}}

\PY{k}{class}\PY{+w}{ }\PY{n+nc}{ProgramDemo}
\PY{p}{\PYZob{}}
\PY{+w}{    }\PY{c+c1}{// Methods for delegate}
\PY{+w}{    }\PY{k}{public}\PY{+w}{ }\PY{k}{static}\PY{+w}{ }\PY{k}{void}\PY{+w}{ }\PY{n+nf}{Method1}\PY{p}{(}\PY{k+kt}{string}\PY{+w}{ }\PY{n}{msg}\PY{p}{)}
\PY{+w}{    }\PY{p}{\PYZob{}}
\PY{+w}{        }\PY{n}{Console}\PY{p}{.}\PY{n}{WriteLine}\PY{p}{(}\PY{l+s}{\PYZdq{}Method1: \PYZdq{}}\PY{+w}{ }\PY{o}{+}\PY{+w}{ }\PY{n}{msg}\PY{p}{)}\PY{p}{;}
\PY{+w}{    }\PY{p}{\PYZcb{}}

\PY{+w}{    }\PY{k}{public}\PY{+w}{ }\PY{k}{static}\PY{+w}{ }\PY{k}{void}\PY{+w}{ }\PY{n+nf}{Method2}\PY{p}{(}\PY{k+kt}{string}\PY{+w}{ }\PY{n}{msg}\PY{p}{)}
\PY{+w}{    }\PY{p}{\PYZob{}}
\PY{+w}{        }\PY{n}{Console}\PY{p}{.}\PY{n}{WriteLine}\PY{p}{(}\PY{l+s}{\PYZdq{}Method2: \PYZdq{}}\PY{+w}{ }\PY{o}{+}\PY{+w}{ }\PY{n}{msg}\PY{p}{)}\PY{p}{;}
\PY{+w}{    }\PY{p}{\PYZcb{}}
\PY{p}{\PYZcb{}}

\PY{c+c1}{// Top\PYZhy{}level code}

\PY{c+c1}{// (a) Delegate}
\PY{n}{MyDelegate}\PY{+w}{ }\PY{n}{d1}\PY{+w}{ }\PY{o}{=}\PY{+w}{ }\PY{n}{ProgramDemo}\PY{p}{.}\PY{n}{Method1}\PY{p}{;}
\PY{n}{d1}\PY{p}{(}\PY{l+s}{\PYZdq{}Simple Delegate\PYZdq{}}\PY{p}{)}\PY{p}{;}

\PY{c+c1}{// (b) Multicast Delegate}
\PY{n}{MyDelegate}\PY{+w}{ }\PY{n}{d2}\PY{+w}{ }\PY{o}{=}\PY{+w}{ }\PY{n}{ProgramDemo}\PY{p}{.}\PY{n}{Method1}\PY{p}{;}
\PY{n}{d2}\PY{+w}{ }\PY{o}{+=}\PY{+w}{ }\PY{n}{ProgramDemo}\PY{p}{.}\PY{n}{Method2}\PY{p}{;}
\PY{n}{Console}\PY{p}{.}\PY{n}{WriteLine}\PY{p}{(}\PY{l+s}{\PYZdq{}\PYZbs{}nMulticast Delegate:\PYZdq{}}\PY{p}{)}\PY{p}{;}
\PY{n}{d2}\PY{p}{(}\PY{l+s}{\PYZdq{}Hello\PYZdq{}}\PY{p}{)}\PY{p}{;}

\PY{c+c1}{// (c) Func Delegate}
\PY{n}{Func}\PY{o}{\PYZlt{}}\PY{k+kt}{int}\PY{p}{,}\PY{+w}{ }\PY{k+kt}{int}\PY{p}{,}\PY{+w}{ }\PY{k+kt}{int}\PY{o}{\PYZgt{}}\PY{+w}{ }\PY{k}{add}\PY{+w}{ }\PY{o}{=}\PY{+w}{ }\PY{p}{(}\PY{n}{a}\PY{p}{,}\PY{+w}{ }\PY{n}{b}\PY{p}{)}\PY{+w}{ }\PY{o}{=\PYZgt{}}\PY{+w}{ }\PY{n}{a}\PY{+w}{ }\PY{o}{+}\PY{+w}{ }\PY{n}{b}\PY{p}{;}
\PY{n}{Console}\PY{p}{.}\PY{n}{WriteLine}\PY{p}{(}\PY{l+s}{\PYZdq{}\PYZbs{}nFunc Delegate Result: \PYZdq{}}\PY{+w}{ }\PY{o}{+}\PY{+w}{ }\PY{k}{add}\PY{p}{(}\PY{l+m+mi}{10}\PY{p}{,}\PY{+w}{ }\PY{l+m+mi}{20}\PY{p}{)}\PY{p}{)}\PY{p}{;}

\PY{c+c1}{// (d) Action Delegate}
\PY{n}{Action}\PY{o}{\PYZlt{}}\PY{k+kt}{string}\PY{o}{\PYZgt{}}\PY{+w}{ }\PY{n}{action}\PY{+w}{ }\PY{o}{=}\PY{+w}{ }\PY{n}{msg}\PY{+w}{ }\PY{o}{=\PYZgt{}}\PY{+w}{ }\PY{n}{Console}\PY{p}{.}\PY{n}{WriteLine}\PY{p}{(}\PY{l+s}{\PYZdq{}Action Delegate: \PYZdq{}}\PY{+w}{ }\PY{o}{+}\PY{+w}{ }\PY{n}{msg}\PY{p}{)}\PY{p}{;}
\PY{n}{action}\PY{p}{(}\PY{l+s}{\PYZdq{}Welcome\PYZdq{}}\PY{p}{)}\PY{p}{;}

\PY{c+c1}{// (e) Anonymous Method}
\PY{n}{MyDelegate}\PY{+w}{ }\PY{n}{anon}\PY{+w}{ }\PY{o}{=}\PY{+w}{ }\PY{k}{delegate}\PY{+w}{ }\PY{p}{(}\PY{k+kt}{string}\PY{+w}{ }\PY{n}{msg}\PY{p}{)}
\PY{p}{\PYZob{}}
\PY{+w}{    }\PY{n}{Console}\PY{p}{.}\PY{n}{WriteLine}\PY{p}{(}\PY{l+s}{\PYZdq{}Anonymous Method: \PYZdq{}}\PY{+w}{ }\PY{o}{+}\PY{+w}{ }\PY{n}{msg}\PY{p}{)}\PY{p}{;}
\PY{p}{\PYZcb{}}\PY{p}{;}
\PY{n}{anon}\PY{p}{(}\PY{l+s}{\PYZdq{}Hello Anonymous\PYZdq{}}\PY{p}{)}\PY{p}{;}

\PY{c+c1}{// (f) Event}
\PY{n}{EventDemo}\PY{+w}{ }\PY{n}{ev}\PY{+w}{ }\PY{o}{=}\PY{+w}{ }\PY{k}{new}\PY{+w}{ }\PY{n}{EventDemo}\PY{p}{(}\PY{p}{)}\PY{p}{;}
\PY{n}{ev}\PY{p}{.}\PY{n}{MyEvent}\PY{+w}{ }\PY{o}{+=}\PY{+w}{ }\PY{n}{ProgramDemo}\PY{p}{.}\PY{n}{Method1}\PY{p}{;}
\PY{n}{Console}\PY{p}{.}\PY{n}{WriteLine}\PY{p}{(}\PY{l+s}{\PYZdq{}\PYZbs{}nEvent Example:\PYZdq{}}\PY{p}{)}\PY{p}{;}
\PY{n}{ev}\PY{p}{.}\PY{n}{RaiseEvent}\PY{p}{(}\PY{p}{)}\PY{p}{;}

\PY{c+c1}{// Student details}
\PY{n}{Console}\PY{p}{.}\PY{n}{WriteLine}\PY{p}{(}\PY{l+s}{\PYZdq{}\PYZbs{}nStudent Details:\PYZdq{}}\PY{p}{)}\PY{p}{;}
\PY{n}{Console}\PY{p}{.}\PY{n}{WriteLine}\PY{p}{(}\PY{l+s}{\PYZdq{}Name: Mohit Tharu\PYZdq{}}\PY{p}{)}\PY{p}{;}
\PY{n}{Console}\PY{p}{.}\PY{n}{WriteLine}\PY{p}{(}\PY{l+s}{\PYZdq{}Roll No: 117\PYZdq{}}\PY{p}{)}\PY{p}{;}
\PY{n}{Console}\PY{p}{.}\PY{n}{WriteLine}\PY{p}{(}\PY{l+s}{\PYZdq{}College: Patan Multiple Campus\PYZdq{}}\PY{p}{)}\PY{p}{;}

\PY{n}{Console}\PY{p}{.}\PY{n}{ReadLine}\PY{p}{(}\PY{p}{)}\PY{p}{;}
\end{Verbatim}
\end{tcolorbox}

    \begin{Verbatim}[commandchars=\\\{\}]
Method1: Simple Delegate

Multicast Delegate:
Method1: Hello
Method2: Hello

Func Delegate Result: 30
Action Delegate: Welcome
Anonymous Method: Hello Anonymous

Event Example:
Method1: Event has been triggered

Student Details:
Name: Mohit Tharu
Roll No: 117
College: Patan Multiple Campus
    \end{Verbatim}

    \hypertarget{wap-which-use-any-a-non-generic-collection-b-generic-collection}{%
\section{10. WAP which use any a) Non generic collection b) Generic
Collection}\label{wap-which-use-any-a-non-generic-collection-b-generic-collection}}

Theory: Non-generic collections: Collections that can store any type of
object. Examples: ArrayList, Hashtable, Queue, Stack. They are not
type-safe. Generic collections: Collections that store specific types.
Examples: List, Dictionary\textless TKey, TValue\textgreater, Queue.
They are type-safe and provide better performance.

    \begin{tcolorbox}[breakable, size=fbox, boxrule=1pt, pad at break*=1mm,colback=cellbackground, colframe=cellborder]
\prompt{In}{incolor}{47}{\boxspacing}
\begin{Verbatim}[commandchars=\\\{\}]
\PY{k}{using}\PY{+w}{ }\PY{n+nn}{System}\PY{p}{;}
\PY{k}{using}\PY{+w}{ }\PY{n+nn}{System.Collections}\PY{p}{;}
\PY{k}{using}\PY{+w}{ }\PY{n+nn}{System.Collections.Generic}\PY{p}{;}

\PY{c+c1}{// Top\PYZhy{}level code}

\PY{c+c1}{// (a) Non\PYZhy{}Generic Collection (ArrayList)}
\PY{n}{ArrayList}\PY{+w}{ }\PY{n}{list}\PY{+w}{ }\PY{o}{=}\PY{+w}{ }\PY{k}{new}\PY{+w}{ }\PY{n}{ArrayList}\PY{p}{(}\PY{p}{)}\PY{p}{;}
\PY{n}{list}\PY{p}{.}\PY{n}{Add}\PY{p}{(}\PY{l+s}{\PYZdq{}Mohit\PYZdq{}}\PY{p}{)}\PY{p}{;}
\PY{n}{list}\PY{p}{.}\PY{n}{Add}\PY{p}{(}\PY{l+m+mi}{117}\PY{p}{)}\PY{p}{;}
\PY{n}{list}\PY{p}{.}\PY{n}{Add}\PY{p}{(}\PY{l+m+mf}{75.5}\PY{p}{)}\PY{p}{;}

\PY{n}{Console}\PY{p}{.}\PY{n}{WriteLine}\PY{p}{(}\PY{l+s}{\PYZdq{}Non\PYZhy{}Generic Collection (ArrayList):\PYZdq{}}\PY{p}{)}\PY{p}{;}
\PY{k}{foreach}\PY{+w}{ }\PY{p}{(}\PY{k+kt}{var}\PY{+w}{ }\PY{n}{item}\PY{+w}{ }\PY{k}{in}\PY{+w}{ }\PY{n}{list}\PY{p}{)}
\PY{p}{\PYZob{}}
\PY{+w}{    }\PY{n}{Console}\PY{p}{.}\PY{n}{WriteLine}\PY{p}{(}\PY{n}{item}\PY{p}{)}\PY{p}{;}
\PY{p}{\PYZcb{}}

\PY{c+c1}{// (b) Generic Collection (List\PYZlt{}T\PYZgt{})}
\PY{n}{List}\PY{o}{\PYZlt{}}\PY{k+kt}{string}\PY{o}{\PYZgt{}}\PY{+w}{ }\PY{n}{names}\PY{+w}{ }\PY{o}{=}\PY{+w}{ }\PY{k}{new}\PY{+w}{ }\PY{n}{List}\PY{o}{\PYZlt{}}\PY{k+kt}{string}\PY{o}{\PYZgt{}}\PY{p}{(}\PY{p}{)}\PY{p}{;}
\PY{n}{names}\PY{p}{.}\PY{n}{Add}\PY{p}{(}\PY{l+s}{\PYZdq{}Mohit Tharu\PYZdq{}}\PY{p}{)}\PY{p}{;}
\PY{n}{names}\PY{p}{.}\PY{n}{Add}\PY{p}{(}\PY{l+s}{\PYZdq{}Rahul\PYZdq{}}\PY{p}{)}\PY{p}{;}
\PY{n}{names}\PY{p}{.}\PY{n}{Add}\PY{p}{(}\PY{l+s}{\PYZdq{}Sita\PYZdq{}}\PY{p}{)}\PY{p}{;}

\PY{n}{Console}\PY{p}{.}\PY{n}{WriteLine}\PY{p}{(}\PY{l+s}{\PYZdq{}\PYZbs{}nGeneric Collection (List\PYZlt{}string\PYZgt{}):\PYZdq{}}\PY{p}{)}\PY{p}{;}
\PY{k}{foreach}\PY{+w}{ }\PY{p}{(}\PY{k+kt}{string}\PY{+w}{ }\PY{n}{name}\PY{+w}{ }\PY{k}{in}\PY{+w}{ }\PY{n}{names}\PY{p}{)}
\PY{p}{\PYZob{}}
\PY{+w}{    }\PY{n}{Console}\PY{p}{.}\PY{n}{WriteLine}\PY{p}{(}\PY{n}{name}\PY{p}{)}\PY{p}{;}
\PY{p}{\PYZcb{}}

\PY{c+c1}{// Student details}
\PY{n}{Console}\PY{p}{.}\PY{n}{WriteLine}\PY{p}{(}\PY{l+s}{\PYZdq{}\PYZbs{}nStudent Details:\PYZdq{}}\PY{p}{)}\PY{p}{;}
\PY{n}{Console}\PY{p}{.}\PY{n}{WriteLine}\PY{p}{(}\PY{l+s}{\PYZdq{}Name: Mohit Tharu\PYZdq{}}\PY{p}{)}\PY{p}{;}
\PY{n}{Console}\PY{p}{.}\PY{n}{WriteLine}\PY{p}{(}\PY{l+s}{\PYZdq{}Roll No: 117\PYZdq{}}\PY{p}{)}\PY{p}{;}
\PY{n}{Console}\PY{p}{.}\PY{n}{WriteLine}\PY{p}{(}\PY{l+s}{\PYZdq{}College: Patan Multiple Campus\PYZdq{}}\PY{p}{)}\PY{p}{;}

\PY{n}{Console}\PY{p}{.}\PY{n}{ReadLine}\PY{p}{(}\PY{p}{)}\PY{p}{;}
\end{Verbatim}
\end{tcolorbox}

    \begin{Verbatim}[commandchars=\\\{\}]
Non-Generic Collection (ArrayList):
Mohit
117
75.5

Generic Collection (List<string>):
Mohit Tharu
Rahul
Sita

Student Details:
Name: Mohit Tharu
Roll No: 117
College: Patan Multiple Campus
    \end{Verbatim}

    \hypertarget{program-to-demonstrate-the-use-of-generic-class-with-generic-field-and-method.}{%
\section{11. Program to demonstrate the use of Generic Class with
Generic field and
method.}\label{program-to-demonstrate-the-use-of-generic-class-with-generic-field-and-method.}}

Theory: Generic Class: A class that can operate on any data type
specified at object creation. Generic Field: A class member whose type
is generic. Generic Method: A method that can work with any data type,
independent of the class's type or other types. Benefits: Type :safety
,Code ,reusability

    \begin{tcolorbox}[breakable, size=fbox, boxrule=1pt, pad at break*=1mm,colback=cellbackground, colframe=cellborder]
\prompt{In}{incolor}{50}{\boxspacing}
\begin{Verbatim}[commandchars=\\\{\}]
\PY{k}{using}\PY{+w}{ }\PY{n+nn}{System}\PY{p}{;}

\PY{c+c1}{// Generic Class}
\PY{k}{class}\PY{+w}{ }\PY{n+nc}{MyGeneric}\PY{o}{\PYZlt{}}\PY{n}{T}\PY{o}{\PYZgt{}}
\PY{p}{\PYZob{}}
\PY{+w}{    }\PY{c+c1}{// Generic Field}
\PY{+w}{    }\PY{k}{public}\PY{+w}{ }\PY{n}{T}\PY{+w}{ }\PY{n}{data}\PY{p}{;}

\PY{+w}{    }\PY{c+c1}{// Constructor}
\PY{+w}{    }\PY{k}{public}\PY{+w}{ }\PY{n+nf}{MyGeneric}\PY{p}{(}\PY{n}{T}\PY{+w}{ }\PY{k}{value}\PY{p}{)}
\PY{+w}{    }\PY{p}{\PYZob{}}
\PY{+w}{        }\PY{n}{data}\PY{+w}{ }\PY{o}{=}\PY{+w}{ }\PY{k}{value}\PY{p}{;}
\PY{+w}{    }\PY{p}{\PYZcb{}}

\PY{+w}{    }\PY{c+c1}{// Generic Method}
\PY{+w}{    }\PY{k}{public}\PY{+w}{ }\PY{k}{void}\PY{+w}{ }\PY{n}{ShowData}\PY{o}{\PYZlt{}}\PY{n}{U}\PY{o}{\PYZgt{}}\PY{p}{(}\PY{n}{U}\PY{+w}{ }\PY{n}{info}\PY{p}{)}
\PY{+w}{    }\PY{p}{\PYZob{}}
\PY{+w}{        }\PY{n}{Console}\PY{p}{.}\PY{n}{WriteLine}\PY{p}{(}\PY{l+s}{\PYZdq{}Generic Field Value: \PYZdq{}}\PY{+w}{ }\PY{o}{+}\PY{+w}{ }\PY{n}{data}\PY{p}{)}\PY{p}{;}
\PY{+w}{        }\PY{n}{Console}\PY{p}{.}\PY{n}{WriteLine}\PY{p}{(}\PY{l+s}{\PYZdq{}Generic Method Value: \PYZdq{}}\PY{+w}{ }\PY{o}{+}\PY{+w}{ }\PY{n}{info}\PY{p}{)}\PY{p}{;}
\PY{+w}{    }\PY{p}{\PYZcb{}}
\PY{p}{\PYZcb{}}

\PY{c+c1}{// Top\PYZhy{}level code}
\PY{n}{MyGeneric}\PY{o}{\PYZlt{}}\PY{k+kt}{int}\PY{o}{\PYZgt{}}\PY{+w}{ }\PY{n}{obj1}\PY{+w}{ }\PY{o}{=}\PY{+w}{ }\PY{k}{new}\PY{+w}{ }\PY{n}{MyGeneric}\PY{o}{\PYZlt{}}\PY{k+kt}{int}\PY{o}{\PYZgt{}}\PY{p}{(}\PY{l+m+mi}{100}\PY{p}{)}\PY{p}{;}
\PY{n}{obj1}\PY{p}{.}\PY{n}{ShowData}\PY{o}{\PYZlt{}}\PY{k+kt}{string}\PY{o}{\PYZgt{}}\PY{p}{(}\PY{l+s}{\PYZdq{}Hello Generic\PYZdq{}}\PY{p}{)}\PY{p}{;}

\PY{n}{MyGeneric}\PY{o}{\PYZlt{}}\PY{k+kt}{string}\PY{o}{\PYZgt{}}\PY{+w}{ }\PY{n}{obj2}\PY{+w}{ }\PY{o}{=}\PY{+w}{ }\PY{k}{new}\PY{+w}{ }\PY{n}{MyGeneric}\PY{o}{\PYZlt{}}\PY{k+kt}{string}\PY{o}{\PYZgt{}}\PY{p}{(}\PY{l+s}{\PYZdq{}Mohit Tharu\PYZdq{}}\PY{p}{)}\PY{p}{;}
\PY{n}{obj2}\PY{p}{.}\PY{n}{ShowData}\PY{o}{\PYZlt{}}\PY{k+kt}{int}\PY{o}{\PYZgt{}}\PY{p}{(}\PY{l+m+mi}{117}\PY{p}{)}\PY{p}{;}

\PY{c+c1}{// Student details}
\PY{n}{Console}\PY{p}{.}\PY{n}{WriteLine}\PY{p}{(}\PY{l+s}{\PYZdq{}\PYZbs{}nStudent Details:\PYZdq{}}\PY{p}{)}\PY{p}{;}
\PY{n}{Console}\PY{p}{.}\PY{n}{WriteLine}\PY{p}{(}\PY{l+s}{\PYZdq{}Name: Mohit Tharu\PYZdq{}}\PY{p}{)}\PY{p}{;}
\PY{n}{Console}\PY{p}{.}\PY{n}{WriteLine}\PY{p}{(}\PY{l+s}{\PYZdq{}Roll No: 117\PYZdq{}}\PY{p}{)}\PY{p}{;}
\PY{n}{Console}\PY{p}{.}\PY{n}{WriteLine}\PY{p}{(}\PY{l+s}{\PYZdq{}College: Patan Multiple Campus\PYZdq{}}\PY{p}{)}\PY{p}{;}

\PY{n}{Console}\PY{p}{.}\PY{n}{ReadLine}\PY{p}{(}\PY{p}{)}\PY{p}{;}
\end{Verbatim}
\end{tcolorbox}

    \begin{Verbatim}[commandchars=\\\{\}]
Generic Field Value: 100
Generic Method Value: Hello Generic
Generic Field Value: Mohit Tharu
Generic Method Value: 117

Student Details:
Name: Mohit Tharu
Roll No: 117
College: Patan Multiple Campus
    \end{Verbatim}

    \hypertarget{wap-to-take-input-from-keyboard-and-write-them-to-a-file}{%
\section{12. WAP to take input from keyboard and write them to a
file}\label{wap-to-take-input-from-keyboard-and-write-them-to-a-file}}

Theory:File Handling: Writing to a file in C\# can be done using classes
from System.IO namespace such as StreamWriter or FileStream. Keyboard
Input: The Console.ReadLine() method reads input from the user.
Combining these, we can read user input and save it to a text file.

    \begin{tcolorbox}[breakable, size=fbox, boxrule=1pt, pad at break*=1mm,colback=cellbackground, colframe=cellborder]
\prompt{In}{incolor}{57}{\boxspacing}
\begin{Verbatim}[commandchars=\\\{\}]
\PY{k}{using}\PY{+w}{ }\PY{n+nn}{System}\PY{p}{;}
\PY{k}{using}\PY{+w}{ }\PY{n+nn}{System.IO}\PY{p}{;}

\PY{c+c1}{// File path}
\PY{k+kt}{string}\PY{+w}{ }\PY{n}{filePath}\PY{+w}{ }\PY{o}{=}\PY{+w}{ }\PY{l+s}{\PYZdq{}studentdata.txt\PYZdq{}}\PY{p}{;}

\PY{c+c1}{// Simulated input (used when keyboard input is not supported)}
\PY{k+kt}{string}\PY{+w}{ }\PY{n}{input}\PY{+w}{ }\PY{o}{=}\PY{+w}{ }\PY{l+s}{\PYZdq{}This is my first file handling program in C\PYZsh{}\PYZdq{}}\PY{p}{;}

\PY{c+c1}{// Writing to file}
\PY{k}{using}\PY{+w}{ }\PY{p}{(}\PY{n}{StreamWriter}\PY{+w}{ }\PY{n}{writer}\PY{+w}{ }\PY{o}{=}\PY{+w}{ }\PY{k}{new}\PY{+w}{ }\PY{n}{StreamWriter}\PY{p}{(}\PY{n}{filePath}\PY{p}{)}\PY{p}{)}
\PY{p}{\PYZob{}}
\PY{+w}{    }\PY{n}{writer}\PY{p}{.}\PY{n}{WriteLine}\PY{p}{(}\PY{l+s}{\PYZdq{}User Input:\PYZdq{}}\PY{p}{)}\PY{p}{;}
\PY{+w}{    }\PY{n}{writer}\PY{p}{.}\PY{n}{WriteLine}\PY{p}{(}\PY{n}{input}\PY{p}{)}\PY{p}{;}
\PY{p}{\PYZcb{}}

\PY{n}{Console}\PY{p}{.}\PY{n}{WriteLine}\PY{p}{(}\PY{l+s}{\PYZdq{}Data written to file successfully!\PYZdq{}}\PY{p}{)}\PY{p}{;}
\PY{n}{Console}\PY{p}{.}\PY{n}{WriteLine}\PY{p}{(}\PY{l+s}{\PYZdq{}Written Content:\PYZdq{}}\PY{p}{)}\PY{p}{;}
\PY{n}{Console}\PY{p}{.}\PY{n}{WriteLine}\PY{p}{(}\PY{n}{input}\PY{p}{)}\PY{p}{;}

\PY{c+c1}{// Student details}
\PY{n}{Console}\PY{p}{.}\PY{n}{WriteLine}\PY{p}{(}\PY{l+s}{\PYZdq{}\PYZbs{}nStudent Details:\PYZdq{}}\PY{p}{)}\PY{p}{;}
\PY{n}{Console}\PY{p}{.}\PY{n}{WriteLine}\PY{p}{(}\PY{l+s}{\PYZdq{}Name: Mohit Tharu\PYZdq{}}\PY{p}{)}\PY{p}{;}
\PY{n}{Console}\PY{p}{.}\PY{n}{WriteLine}\PY{p}{(}\PY{l+s}{\PYZdq{}Roll No: 117\PYZdq{}}\PY{p}{)}\PY{p}{;}
\PY{n}{Console}\PY{p}{.}\PY{n}{WriteLine}\PY{p}{(}\PY{l+s}{\PYZdq{}College: Patan Multiple Campus\PYZdq{}}\PY{p}{)}\PY{p}{;}

\PY{n}{Console}\PY{p}{.}\PY{n}{ReadLine}\PY{p}{(}\PY{p}{)}\PY{p}{;}
\end{Verbatim}
\end{tcolorbox}

    \begin{Verbatim}[commandchars=\\\{\}]
Data written to file successfully!
Written Content:
This is my first file handling program in C\#

Student Details:
Name: Mohit Tharu
Roll No: 117
College: Patan Multiple Campus
    \end{Verbatim}

    \hypertarget{wap-to-demonstrate-the-concept-of-linq}{%
\section{13. WAP to demonstrate the concept of
LINQ}\label{wap-to-demonstrate-the-concept-of-linq}}

Theory: LINQ allows querying collections (arrays, lists, XML, databases)
directly in C\# using query syntax or method syntax. Benefits:
-\textgreater Simplifies data querying -\textgreater Strongly typed,
compile-time checking -\textgreater Works on different data source.

    \begin{tcolorbox}[breakable, size=fbox, boxrule=1pt, pad at break*=1mm,colback=cellbackground, colframe=cellborder]
\prompt{In}{incolor}{62}{\boxspacing}
\begin{Verbatim}[commandchars=\\\{\}]
\PY{k}{using}\PY{+w}{ }\PY{n+nn}{System}\PY{p}{;}
\PY{k}{using}\PY{+w}{ }\PY{n+nn}{System.Linq}\PY{p}{;}
\PY{k}{using}\PY{+w}{ }\PY{n+nn}{System.Collections.Generic}\PY{p}{;}

\PY{c+c1}{// Data source}
\PY{n}{List}\PY{o}{\PYZlt{}}\PY{k+kt}{int}\PY{o}{\PYZgt{}}\PY{+w}{ }\PY{n}{numbers}\PY{+w}{ }\PY{o}{=}\PY{+w}{ }\PY{k}{new}\PY{+w}{ }\PY{n}{List}\PY{o}{\PYZlt{}}\PY{k+kt}{int}\PY{o}{\PYZgt{}}\PY{+w}{ }\PY{p}{\PYZob{}}\PY{+w}{ }\PY{l+m+mi}{10}\PY{p}{,}\PY{+w}{ }\PY{l+m+mi}{25}\PY{p}{,}\PY{+w}{ }\PY{l+m+mi}{30}\PY{p}{,}\PY{+w}{ }\PY{l+m+mi}{45}\PY{p}{,}\PY{+w}{ }\PY{l+m+mi}{60}\PY{p}{,}\PY{+w}{ }\PY{l+m+mi}{75}\PY{+w}{ }\PY{p}{\PYZcb{}}\PY{p}{;}

\PY{c+c1}{// LINQ Query Syntax}
\PY{k+kt}{var}\PY{+w}{ }\PY{n}{evenNumbers}\PY{+w}{ }\PY{o}{=}\PY{+w}{ }\PY{k}{from}\PY{+w}{ }\PY{n}{num}\PY{+w}{ }\PY{k}{in}\PY{+w}{ }\PY{n}{numbers}
\PY{+w}{                  }\PY{k}{where}\PY{+w}{ }\PY{n}{num}\PY{+w}{ }\PY{o}{\PYZpc{}}\PY{+w}{ }\PY{l+m+mi}{2}\PY{+w}{ }\PY{o}{==}\PY{+w}{ }\PY{l+m+mi}{0}
\PY{+w}{                  }\PY{k}{select}\PY{+w}{ }\PY{n}{num}\PY{p}{;}

\PY{c+c1}{// LINQ Method Syntax}
\PY{k+kt}{var}\PY{+w}{ }\PY{n}{greaterThanThirty}\PY{+w}{ }\PY{o}{=}\PY{+w}{ }\PY{n}{numbers}\PY{p}{.}\PY{n}{Where}\PY{p}{(}\PY{n}{n}\PY{+w}{ }\PY{o}{=\PYZgt{}}\PY{+w}{ }\PY{n}{n}\PY{+w}{ }\PY{o}{\PYZgt{}}\PY{+w}{ }\PY{l+m+mi}{30}\PY{p}{)}\PY{p}{;}

\PY{n}{Console}\PY{p}{.}\PY{n}{WriteLine}\PY{p}{(}\PY{l+s}{\PYZdq{}LINQ Query Syntax (Even Numbers):\PYZdq{}}\PY{p}{)}\PY{p}{;}
\PY{k}{foreach}\PY{+w}{ }\PY{p}{(}\PY{k+kt}{var}\PY{+w}{ }\PY{n}{n}\PY{+w}{ }\PY{k}{in}\PY{+w}{ }\PY{n}{evenNumbers}\PY{p}{)}
\PY{p}{\PYZob{}}
\PY{+w}{    }\PY{n}{Console}\PY{p}{.}\PY{n}{WriteLine}\PY{p}{(}\PY{n}{n}\PY{p}{)}\PY{p}{;}
\PY{p}{\PYZcb{}}

\PY{n}{Console}\PY{p}{.}\PY{n}{WriteLine}\PY{p}{(}\PY{l+s}{\PYZdq{}\PYZbs{}nLINQ Method Syntax (Numbers \PYZgt{} 30):\PYZdq{}}\PY{p}{)}\PY{p}{;}
\PY{k}{foreach}\PY{+w}{ }\PY{p}{(}\PY{k+kt}{var}\PY{+w}{ }\PY{n}{n}\PY{+w}{ }\PY{k}{in}\PY{+w}{ }\PY{n}{greaterThanThirty}\PY{p}{)}
\PY{p}{\PYZob{}}
\PY{+w}{    }\PY{n}{Console}\PY{p}{.}\PY{n}{WriteLine}\PY{p}{(}\PY{n}{n}\PY{p}{)}\PY{p}{;}
\PY{p}{\PYZcb{}}

\PY{c+c1}{// Student details}
\PY{n}{Console}\PY{p}{.}\PY{n}{WriteLine}\PY{p}{(}\PY{l+s}{\PYZdq{}\PYZbs{}nStudent Details:\PYZdq{}}\PY{p}{)}\PY{p}{;}
\PY{n}{Console}\PY{p}{.}\PY{n}{WriteLine}\PY{p}{(}\PY{l+s}{\PYZdq{}Name: Mohit Tharu\PYZdq{}}\PY{p}{)}\PY{p}{;}
\PY{n}{Console}\PY{p}{.}\PY{n}{WriteLine}\PY{p}{(}\PY{l+s}{\PYZdq{}Roll No: 117\PYZdq{}}\PY{p}{)}\PY{p}{;}
\PY{n}{Console}\PY{p}{.}\PY{n}{WriteLine}\PY{p}{(}\PY{l+s}{\PYZdq{}College: Patan Multiple Campus\PYZdq{}}\PY{p}{)}\PY{p}{;}

\PY{n}{Console}\PY{p}{.}\PY{n}{ReadLine}\PY{p}{(}\PY{p}{)}\PY{p}{;}
\end{Verbatim}
\end{tcolorbox}

    \begin{Verbatim}[commandchars=\\\{\}]
LINQ Query Syntax (Even Numbers):
10
30
60

LINQ Method Syntax (Numbers > 30):
45
60
75

Student Details:
Name: Mohit Tharu
Roll No: 117
College: Patan Multiple Campus
    \end{Verbatim}

    \hypertarget{wap-to-demonstrate-lamda-expressions-in-c.}{%
\section{14. WAP to demonstrate Lamda Expressions in
C\#.}\label{wap-to-demonstrate-lamda-expressions-in-c.}}

Theory: Lambda Expression is an anonymous function used to create
delegates or expression tree types. Syntax: (parameters) =\textgreater{}
expression Often used with LINQ or generic delegates like Func and
Action

    \begin{tcolorbox}[breakable, size=fbox, boxrule=1pt, pad at break*=1mm,colback=cellbackground, colframe=cellborder]
\prompt{In}{incolor}{65}{\boxspacing}
\begin{Verbatim}[commandchars=\\\{\}]
\PY{k}{using}\PY{+w}{ }\PY{n+nn}{System}\PY{p}{;}
\PY{k}{using}\PY{+w}{ }\PY{n+nn}{System.Collections.Generic}\PY{p}{;}
\PY{k}{using}\PY{+w}{ }\PY{n+nn}{System.Linq}\PY{p}{;}

\PY{c+c1}{// Data source}
\PY{n}{List}\PY{o}{\PYZlt{}}\PY{k+kt}{int}\PY{o}{\PYZgt{}}\PY{+w}{ }\PY{n}{numbers}\PY{+w}{ }\PY{o}{=}\PY{+w}{ }\PY{k}{new}\PY{+w}{ }\PY{n}{List}\PY{o}{\PYZlt{}}\PY{k+kt}{int}\PY{o}{\PYZgt{}}\PY{+w}{ }\PY{p}{\PYZob{}}\PY{+w}{ }\PY{l+m+mi}{5}\PY{p}{,}\PY{+w}{ }\PY{l+m+mi}{10}\PY{p}{,}\PY{+w}{ }\PY{l+m+mi}{15}\PY{p}{,}\PY{+w}{ }\PY{l+m+mi}{20}\PY{p}{,}\PY{+w}{ }\PY{l+m+mi}{25}\PY{p}{,}\PY{+w}{ }\PY{l+m+mi}{30}\PY{+w}{ }\PY{p}{\PYZcb{}}\PY{p}{;}

\PY{c+c1}{// Lambda Expression with Where()}
\PY{k+kt}{var}\PY{+w}{ }\PY{n}{evenNumbers}\PY{+w}{ }\PY{o}{=}\PY{+w}{ }\PY{n}{numbers}\PY{p}{.}\PY{n}{Where}\PY{p}{(}\PY{n}{n}\PY{+w}{ }\PY{o}{=\PYZgt{}}\PY{+w}{ }\PY{n}{n}\PY{+w}{ }\PY{o}{\PYZpc{}}\PY{+w}{ }\PY{l+m+mi}{2}\PY{+w}{ }\PY{o}{==}\PY{+w}{ }\PY{l+m+mi}{0}\PY{p}{)}\PY{p}{;}

\PY{c+c1}{// Lambda Expression with Select()}
\PY{k+kt}{var}\PY{+w}{ }\PY{n}{squares}\PY{+w}{ }\PY{o}{=}\PY{+w}{ }\PY{n}{numbers}\PY{p}{.}\PY{n}{Select}\PY{p}{(}\PY{n}{n}\PY{+w}{ }\PY{o}{=\PYZgt{}}\PY{+w}{ }\PY{n}{n}\PY{+w}{ }\PY{o}{*}\PY{+w}{ }\PY{n}{n}\PY{p}{)}\PY{p}{;}

\PY{n}{Console}\PY{p}{.}\PY{n}{WriteLine}\PY{p}{(}\PY{l+s}{\PYZdq{}Even Numbers using Lambda Expression:\PYZdq{}}\PY{p}{)}\PY{p}{;}
\PY{k}{foreach}\PY{+w}{ }\PY{p}{(}\PY{k+kt}{var}\PY{+w}{ }\PY{n}{n}\PY{+w}{ }\PY{k}{in}\PY{+w}{ }\PY{n}{evenNumbers}\PY{p}{)}
\PY{p}{\PYZob{}}
\PY{+w}{    }\PY{n}{Console}\PY{p}{.}\PY{n}{WriteLine}\PY{p}{(}\PY{n}{n}\PY{p}{)}\PY{p}{;}
\PY{p}{\PYZcb{}}

\PY{n}{Console}\PY{p}{.}\PY{n}{WriteLine}\PY{p}{(}\PY{l+s}{\PYZdq{}\PYZbs{}nSquares using Lambda Expression:\PYZdq{}}\PY{p}{)}\PY{p}{;}
\PY{k}{foreach}\PY{+w}{ }\PY{p}{(}\PY{k+kt}{var}\PY{+w}{ }\PY{n}{s}\PY{+w}{ }\PY{k}{in}\PY{+w}{ }\PY{n}{squares}\PY{p}{)}
\PY{p}{\PYZob{}}
\PY{+w}{    }\PY{n}{Console}\PY{p}{.}\PY{n}{WriteLine}\PY{p}{(}\PY{n}{s}\PY{p}{)}\PY{p}{;}
\PY{p}{\PYZcb{}}

\PY{c+c1}{// Lambda Expression with Action}
\PY{n}{Action}\PY{o}{\PYZlt{}}\PY{k+kt}{string}\PY{o}{\PYZgt{}}\PY{+w}{ }\PY{n}{greet}\PY{+w}{ }\PY{o}{=}\PY{+w}{ }\PY{n}{msg}\PY{+w}{ }\PY{o}{=\PYZgt{}}\PY{+w}{ }\PY{n}{Console}\PY{p}{.}\PY{n}{WriteLine}\PY{p}{(}\PY{l+s}{\PYZdq{}Message: \PYZdq{}}\PY{+w}{ }\PY{o}{+}\PY{+w}{ }\PY{n}{msg}\PY{p}{)}\PY{p}{;}
\PY{n}{greet}\PY{p}{(}\PY{l+s}{\PYZdq{}Hello from Lambda Expression\PYZdq{}}\PY{p}{)}\PY{p}{;}

\PY{c+c1}{// Student details}
\PY{n}{Console}\PY{p}{.}\PY{n}{WriteLine}\PY{p}{(}\PY{l+s}{\PYZdq{}\PYZbs{}nStudent Details:\PYZdq{}}\PY{p}{)}\PY{p}{;}
\PY{n}{Console}\PY{p}{.}\PY{n}{WriteLine}\PY{p}{(}\PY{l+s}{\PYZdq{}Name: Mohit Tharu\PYZdq{}}\PY{p}{)}\PY{p}{;}
\PY{n}{Console}\PY{p}{.}\PY{n}{WriteLine}\PY{p}{(}\PY{l+s}{\PYZdq{}Roll No: 117\PYZdq{}}\PY{p}{)}\PY{p}{;}
\PY{n}{Console}\PY{p}{.}\PY{n}{WriteLine}\PY{p}{(}\PY{l+s}{\PYZdq{}College: Patan Multiple Campus\PYZdq{}}\PY{p}{)}\PY{p}{;}

\PY{n}{Console}\PY{p}{.}\PY{n}{ReadLine}\PY{p}{(}\PY{p}{)}\PY{p}{;}
\end{Verbatim}
\end{tcolorbox}

    \begin{Verbatim}[commandchars=\\\{\}]
Even Numbers using Lambda Expression:
10
20
30

Squares using Lambda Expression:
25
100
225
400
625
900
Message: Hello from Lambda Expression

Student Details:
Name: Mohit Tharu
Roll No: 117
College: Patan Multiple Campus
    \end{Verbatim}

    \hypertarget{wap-to}{%
\section{15. WAP to}\label{wap-to}}

\begin{verbatim}
# a) demonstrate exception handling in C# using try, catch and finally blocks.
# b) deal with throw keyword in exception handling 
# c) demonstrate custom exception handling
\end{verbatim}

Theory: Exception Handling: Mechanism to handle runtime errors without
crashing the program. try block: Code that may generate exception is
placed here.

    \begin{tcolorbox}[breakable, size=fbox, boxrule=1pt, pad at break*=1mm,colback=cellbackground, colframe=cellborder]
\prompt{In}{incolor}{68}{\boxspacing}
\begin{Verbatim}[commandchars=\\\{\}]
\PY{k}{using}\PY{+w}{ }\PY{n+nn}{System}\PY{p}{;}

\PY{c+c1}{// (c) Custom Exception}
\PY{k}{class}\PY{+w}{ }\PY{n+nc}{AgeException}\PY{+w}{ }\PY{p}{:}\PY{+w}{ }\PY{n}{Exception}
\PY{p}{\PYZob{}}
\PY{+w}{    }\PY{k}{public}\PY{+w}{ }\PY{n+nf}{AgeException}\PY{p}{(}\PY{k+kt}{string}\PY{+w}{ }\PY{n}{message}\PY{p}{)}\PY{+w}{ }\PY{p}{:}\PY{+w}{ }\PY{k}{base}\PY{p}{(}\PY{n}{message}\PY{p}{)}
\PY{+w}{    }\PY{p}{\PYZob{}}
\PY{+w}{    }\PY{p}{\PYZcb{}}
\PY{p}{\PYZcb{}}

\PY{k}{class}\PY{+w}{ }\PY{n+nc}{ExceptionDemo}
\PY{p}{\PYZob{}}
\PY{+w}{    }\PY{c+c1}{// Method using throw keyword}
\PY{+w}{    }\PY{k}{public}\PY{+w}{ }\PY{k}{static}\PY{+w}{ }\PY{k}{void}\PY{+w}{ }\PY{n+nf}{CheckAge}\PY{p}{(}\PY{k+kt}{int}\PY{+w}{ }\PY{n}{age}\PY{p}{)}
\PY{+w}{    }\PY{p}{\PYZob{}}
\PY{+w}{        }\PY{k}{if}\PY{+w}{ }\PY{p}{(}\PY{n}{age}\PY{+w}{ }\PY{o}{\PYZlt{}}\PY{+w}{ }\PY{l+m+mi}{18}\PY{p}{)}
\PY{+w}{        }\PY{p}{\PYZob{}}
\PY{+w}{            }\PY{c+c1}{// (b) throw keyword}
\PY{+w}{            }\PY{k}{throw}\PY{+w}{ }\PY{k}{new}\PY{+w}{ }\PY{n+nf}{AgeException}\PY{p}{(}\PY{l+s}{\PYZdq{}Age must be 18 or above.\PYZdq{}}\PY{p}{)}\PY{p}{;}
\PY{+w}{        }\PY{p}{\PYZcb{}}
\PY{+w}{        }\PY{k}{else}
\PY{+w}{        }\PY{p}{\PYZob{}}
\PY{+w}{            }\PY{n}{Console}\PY{p}{.}\PY{n}{WriteLine}\PY{p}{(}\PY{l+s}{\PYZdq{}Valid Age: \PYZdq{}}\PY{+w}{ }\PY{o}{+}\PY{+w}{ }\PY{n}{age}\PY{p}{)}\PY{p}{;}
\PY{+w}{        }\PY{p}{\PYZcb{}}
\PY{+w}{    }\PY{p}{\PYZcb{}}
\PY{p}{\PYZcb{}}

\PY{c+c1}{// Top\PYZhy{}level code}
\PY{k}{try}
\PY{p}{\PYZob{}}
\PY{+w}{    }\PY{c+c1}{// (a) try\PYZhy{}catch\PYZhy{}finally}
\PY{+w}{    }\PY{n}{Console}\PY{p}{.}\PY{n}{WriteLine}\PY{p}{(}\PY{l+s}{\PYZdq{}Enter your age:\PYZdq{}}\PY{p}{)}\PY{p}{;}
\PY{+w}{    }\PY{k+kt}{int}\PY{+w}{ }\PY{n}{age}\PY{+w}{ }\PY{o}{=}\PY{+w}{ }\PY{l+m+mi}{16}\PY{p}{;}\PY{+w}{   }\PY{c+c1}{// fixed value for online compiler}

\PY{+w}{    }\PY{n}{ExceptionDemo}\PY{p}{.}\PY{n}{CheckAge}\PY{p}{(}\PY{n}{age}\PY{p}{)}\PY{p}{;}
\PY{p}{\PYZcb{}}
\PY{k}{catch}\PY{+w}{ }\PY{p}{(}\PY{n}{AgeException}\PY{+w}{ }\PY{n}{ex}\PY{p}{)}
\PY{p}{\PYZob{}}
\PY{+w}{    }\PY{n}{Console}\PY{p}{.}\PY{n}{WriteLine}\PY{p}{(}\PY{l+s}{\PYZdq{}Custom Exception Caught: \PYZdq{}}\PY{+w}{ }\PY{o}{+}\PY{+w}{ }\PY{n}{ex}\PY{p}{.}\PY{n}{Message}\PY{p}{)}\PY{p}{;}
\PY{p}{\PYZcb{}}
\PY{k}{catch}\PY{+w}{ }\PY{p}{(}\PY{n}{Exception}\PY{+w}{ }\PY{n}{ex}\PY{p}{)}
\PY{p}{\PYZob{}}
\PY{+w}{    }\PY{n}{Console}\PY{p}{.}\PY{n}{WriteLine}\PY{p}{(}\PY{l+s}{\PYZdq{}General Exception: \PYZdq{}}\PY{+w}{ }\PY{o}{+}\PY{+w}{ }\PY{n}{ex}\PY{p}{.}\PY{n}{Message}\PY{p}{)}\PY{p}{;}
\PY{p}{\PYZcb{}}
\PY{k}{finally}
\PY{p}{\PYZob{}}
\PY{+w}{    }\PY{n}{Console}\PY{p}{.}\PY{n}{WriteLine}\PY{p}{(}\PY{l+s}{\PYZdq{}Finally block executed\PYZdq{}}\PY{p}{)}\PY{p}{;}
\PY{p}{\PYZcb{}}

\PY{c+c1}{// Student details}
\PY{n}{Console}\PY{p}{.}\PY{n}{WriteLine}\PY{p}{(}\PY{l+s}{\PYZdq{}\PYZbs{}nStudent Details:\PYZdq{}}\PY{p}{)}\PY{p}{;}
\PY{n}{Console}\PY{p}{.}\PY{n}{WriteLine}\PY{p}{(}\PY{l+s}{\PYZdq{}Name: Mohit Tharu\PYZdq{}}\PY{p}{)}\PY{p}{;}
\PY{n}{Console}\PY{p}{.}\PY{n}{WriteLine}\PY{p}{(}\PY{l+s}{\PYZdq{}Roll No: 117\PYZdq{}}\PY{p}{)}\PY{p}{;}
\PY{n}{Console}\PY{p}{.}\PY{n}{WriteLine}\PY{p}{(}\PY{l+s}{\PYZdq{}College: Patan Multiple Campus\PYZdq{}}\PY{p}{)}\PY{p}{;}

\PY{n}{Console}\PY{p}{.}\PY{n}{ReadLine}\PY{p}{(}\PY{p}{)}\PY{p}{;}
\end{Verbatim}
\end{tcolorbox}

    \begin{Verbatim}[commandchars=\\\{\}]
Enter your age:
Custom Exception Caught: Age must be 18 or above.
Finally block executed

Student Details:
Name: Mohit Tharu
Roll No: 117
College: Patan Multiple Campus
    \end{Verbatim}

    \hypertarget{wap-a-to-use-built-in-attributes-in-c.-b-to-create-and-use-custom-attribute-in-c.}{%
\section{16. WAP a) to use built-in attributes in C\#. b) to create and
use custom attribute in
C\#.}\label{wap-a-to-use-built-in-attributes-in-c.-b-to-create-and-use-custom-attribute-in-c.}}

Theory: Attributes: Metadata added to classes, methods, or properties to
provide extra information. Built-in attributes: Predefined by .NET
(e.g., {[}Obsolete{]}, {[}Serializable{]}). Custom attributes:
User-defined attributes by inheriting from System.Attribute.

    \begin{tcolorbox}[breakable, size=fbox, boxrule=1pt, pad at break*=1mm,colback=cellbackground, colframe=cellborder]
\prompt{In}{incolor}{73}{\boxspacing}
\begin{Verbatim}[commandchars=\\\{\}]
\PY{k}{using}\PY{+w}{ }\PY{n+nn}{System}\PY{p}{;}
\PY{k}{using}\PY{+w}{ }\PY{n+nn}{System.Reflection}\PY{p}{;}

\PY{c+c1}{// (b) Custom Attribute}
\PY{n+na}{[AttributeUsage(AttributeTargets.Class | AttributeTargets.Method)]}
\PY{k}{class}\PY{+w}{ }\PY{n+nc}{InfoAttribute}\PY{+w}{ }\PY{p}{:}\PY{+w}{ }\PY{n}{Attribute}
\PY{p}{\PYZob{}}
\PY{+w}{    }\PY{k}{public}\PY{+w}{ }\PY{k+kt}{string}\PY{+w}{ }\PY{n}{Author}\PY{+w}{ }\PY{p}{\PYZob{}}\PY{+w}{ }\PY{k}{get}\PY{p}{;}\PY{+w}{ }\PY{p}{\PYZcb{}}
\PY{+w}{    }\PY{k}{public}\PY{+w}{ }\PY{k+kt}{string}\PY{+w}{ }\PY{n}{Version}\PY{+w}{ }\PY{p}{\PYZob{}}\PY{+w}{ }\PY{k}{get}\PY{p}{;}\PY{+w}{ }\PY{p}{\PYZcb{}}

\PY{+w}{    }\PY{k}{public}\PY{+w}{ }\PY{n+nf}{InfoAttribute}\PY{p}{(}\PY{k+kt}{string}\PY{+w}{ }\PY{n}{author}\PY{p}{,}\PY{+w}{ }\PY{k+kt}{string}\PY{+w}{ }\PY{n}{version}\PY{p}{)}
\PY{+w}{    }\PY{p}{\PYZob{}}
\PY{+w}{        }\PY{n}{Author}\PY{+w}{ }\PY{o}{=}\PY{+w}{ }\PY{n}{author}\PY{p}{;}
\PY{+w}{        }\PY{n}{Version}\PY{+w}{ }\PY{o}{=}\PY{+w}{ }\PY{n}{version}\PY{p}{;}
\PY{+w}{    }\PY{p}{\PYZcb{}}
\PY{p}{\PYZcb{}}

\PY{c+c1}{// Using Built\PYZhy{}in and Custom Attributes}
\PY{n+na}{[Obsolete(\PYZdq{}This class is obsolete. Use NewDemo instead.\PYZdq{})]}
\PY{n+na}{[Info(\PYZdq{}Mohit Tharu\PYZdq{}, \PYZdq{}1.0\PYZdq{})]}
\PY{k}{class}\PY{+w}{ }\PY{n+nc}{Demo}
\PY{p}{\PYZob{}}
\PY{+w}{    }\PY{n+na}{[Info(\PYZdq{}Mohit Tharu\PYZdq{}, \PYZdq{}1.1\PYZdq{})]}
\PY{+w}{    }\PY{k}{public}\PY{+w}{ }\PY{k}{void}\PY{+w}{ }\PY{n+nf}{Show}\PY{p}{(}\PY{p}{)}
\PY{+w}{    }\PY{p}{\PYZob{}}
\PY{+w}{        }\PY{n}{Console}\PY{p}{.}\PY{n}{WriteLine}\PY{p}{(}\PY{l+s}{\PYZdq{}Demo class Show() method\PYZdq{}}\PY{p}{)}\PY{p}{;}
\PY{+w}{    }\PY{p}{\PYZcb{}}
\PY{p}{\PYZcb{}}

\PY{c+c1}{// Top\PYZhy{}level code}
\PY{n}{Demo}\PY{+w}{ }\PY{n}{d}\PY{+w}{ }\PY{o}{=}\PY{+w}{ }\PY{k}{new}\PY{+w}{ }\PY{n}{Demo}\PY{p}{(}\PY{p}{)}\PY{p}{;}
\PY{n}{d}\PY{p}{.}\PY{n}{Show}\PY{p}{(}\PY{p}{)}\PY{p}{;}

\PY{c+c1}{// Reading Custom Attribute using Reflection}
\PY{n}{Type}\PY{+w}{ }\PY{n}{type}\PY{+w}{ }\PY{o}{=}\PY{+w}{ }\PY{k}{typeof}\PY{p}{(}\PY{n}{Demo}\PY{p}{)}\PY{p}{;}
\PY{k+kt}{object}\PY{p}{[}\PY{p}{]}\PY{+w}{ }\PY{n}{attributes}\PY{+w}{ }\PY{o}{=}\PY{+w}{ }\PY{n}{type}\PY{p}{.}\PY{n}{GetCustomAttributes}\PY{p}{(}\PY{k}{typeof}\PY{p}{(}\PY{n}{InfoAttribute}\PY{p}{)}\PY{p}{,}\PY{+w}{ }\PY{k}{false}\PY{p}{)}\PY{p}{;}

\PY{k}{foreach}\PY{+w}{ }\PY{p}{(}\PY{n}{InfoAttribute}\PY{+w}{ }\PY{n}{attr}\PY{+w}{ }\PY{k}{in}\PY{+w}{ }\PY{n}{attributes}\PY{p}{)}
\PY{p}{\PYZob{}}
\PY{+w}{    }\PY{n}{Console}\PY{p}{.}\PY{n}{WriteLine}\PY{p}{(}\PY{l+s}{\PYZdq{}\PYZbs{}nCustom Attribute Details:\PYZdq{}}\PY{p}{)}\PY{p}{;}
\PY{+w}{    }\PY{n}{Console}\PY{p}{.}\PY{n}{WriteLine}\PY{p}{(}\PY{l+s}{\PYZdq{}Author: \PYZdq{}}\PY{+w}{ }\PY{o}{+}\PY{+w}{ }\PY{n}{attr}\PY{p}{.}\PY{n}{Author}\PY{p}{)}\PY{p}{;}
\PY{+w}{    }\PY{n}{Console}\PY{p}{.}\PY{n}{WriteLine}\PY{p}{(}\PY{l+s}{\PYZdq{}Version: \PYZdq{}}\PY{+w}{ }\PY{o}{+}\PY{+w}{ }\PY{n}{attr}\PY{p}{.}\PY{n}{Version}\PY{p}{)}\PY{p}{;}
\PY{p}{\PYZcb{}}

\PY{c+c1}{// Student details}
\PY{n}{Console}\PY{p}{.}\PY{n}{WriteLine}\PY{p}{(}\PY{l+s}{\PYZdq{}\PYZbs{}nStudent Details:\PYZdq{}}\PY{p}{)}\PY{p}{;}
\PY{n}{Console}\PY{p}{.}\PY{n}{WriteLine}\PY{p}{(}\PY{l+s}{\PYZdq{}Name: Mohit Tharu\PYZdq{}}\PY{p}{)}\PY{p}{;}
\PY{n}{Console}\PY{p}{.}\PY{n}{WriteLine}\PY{p}{(}\PY{l+s}{\PYZdq{}Roll No: 117\PYZdq{}}\PY{p}{)}\PY{p}{;}
\PY{n}{Console}\PY{p}{.}\PY{n}{WriteLine}\PY{p}{(}\PY{l+s}{\PYZdq{}College: Patan Multiple Campus\PYZdq{}}\PY{p}{)}\PY{p}{;}

\PY{n}{Console}\PY{p}{.}\PY{n}{ReadLine}\PY{p}{(}\PY{p}{)}\PY{p}{;}
\end{Verbatim}
\end{tcolorbox}

    \begin{Verbatim}[commandchars=\\\{\}]
Demo class Show() method

Custom Attribute Details:
Author: Mohit Tharu
Version: 1.0

Student Details:
Name: Mohit Tharu
Roll No: 117
College: Patan Multiple Campus
    \end{Verbatim}

    \begin{Verbatim}[commandchars=\\\{\}]

(31,1): warning CS0618: 'Demo' is obsolete: 'This class is obsolete. Use NewDemo
instead.'

(31,14): warning CS0618: 'Demo' is obsolete: 'This class is obsolete. Use
NewDemo instead.'

(35,20): warning CS0618: 'Demo' is obsolete: 'This class is obsolete. Use
NewDemo instead.'

    \end{Verbatim}

    \hypertarget{wap-to-demonstrate-asynchronous-programming-in-c-using-async-and-await-keywords.}{%
\section{17. WAP to demonstrate asynchronous programming in C\# using
async and await
keywords.}\label{wap-to-demonstrate-asynchronous-programming-in-c-using-async-and-await-keywords.}}

Theory: Asynchronous programming allows a program to perform tasks
without blocking the main thread. async keyword: Declares a method as
asynchronous. await keyword: Waits for the completion of an asynchronous
task without blocking the main thread. Benefits: Improved responsiveness
in UI applications Efficient use of system resources Better performance
for I/O-bound tasks

    \begin{tcolorbox}[breakable, size=fbox, boxrule=1pt, pad at break*=1mm,colback=cellbackground, colframe=cellborder]
\prompt{In}{incolor}{78}{\boxspacing}
\begin{Verbatim}[commandchars=\\\{\}]
\PY{k}{using}\PY{+w}{ }\PY{n+nn}{System}\PY{p}{;}
\PY{k}{using}\PY{+w}{ }\PY{n+nn}{System.Threading.Tasks}\PY{p}{;}

\PY{k}{class}\PY{+w}{ }\PY{n+nc}{AsyncDemo}
\PY{p}{\PYZob{}}
\PY{+w}{    }\PY{c+c1}{// Asynchronous method}
\PY{+w}{    }\PY{k}{public}\PY{+w}{ }\PY{k}{static}\PY{+w}{ }\PY{k}{async}\PY{+w}{ }\PY{n}{Task}\PY{+w}{ }\PY{n+nf}{LongTask}\PY{p}{(}\PY{p}{)}
\PY{+w}{    }\PY{p}{\PYZob{}}
\PY{+w}{        }\PY{n}{Console}\PY{p}{.}\PY{n}{WriteLine}\PY{p}{(}\PY{l+s}{\PYZdq{}Long task started...\PYZdq{}}\PY{p}{)}\PY{p}{;}
\PY{+w}{        }\PY{k}{await}\PY{+w}{ }\PY{n}{Task}\PY{p}{.}\PY{n}{Delay}\PY{p}{(}\PY{l+m+mi}{3000}\PY{p}{)}\PY{p}{;}\PY{+w}{   }\PY{c+c1}{// Simulates time\PYZhy{}consuming work}
\PY{+w}{        }\PY{n}{Console}\PY{p}{.}\PY{n}{WriteLine}\PY{p}{(}\PY{l+s}{\PYZdq{}Long task completed.\PYZdq{}}\PY{p}{)}\PY{p}{;}
\PY{+w}{    }\PY{p}{\PYZcb{}}
\PY{p}{\PYZcb{}}

\PY{c+c1}{// Top\PYZhy{}level code}
\PY{n}{Console}\PY{p}{.}\PY{n}{WriteLine}\PY{p}{(}\PY{l+s}{\PYZdq{}Program started\PYZdq{}}\PY{p}{)}\PY{p}{;}

\PY{c+c1}{// Calling async method}
\PY{n}{Task}\PY{+w}{ }\PY{n}{t}\PY{+w}{ }\PY{o}{=}\PY{+w}{ }\PY{n}{AsyncDemo}\PY{p}{.}\PY{n}{LongTask}\PY{p}{(}\PY{p}{)}\PY{p}{;}

\PY{c+c1}{// Other work while async task is running}
\PY{n}{Console}\PY{p}{.}\PY{n}{WriteLine}\PY{p}{(}\PY{l+s}{\PYZdq{}Doing other work...\PYZdq{}}\PY{p}{)}\PY{p}{;}

\PY{c+c1}{// Wait for async task to complete}
\PY{n}{t}\PY{p}{.}\PY{n}{Wait}\PY{p}{(}\PY{p}{)}\PY{p}{;}

\PY{n}{Console}\PY{p}{.}\PY{n}{WriteLine}\PY{p}{(}\PY{l+s}{\PYZdq{}Program ended\PYZdq{}}\PY{p}{)}\PY{p}{;}

\PY{c+c1}{// Student details}
\PY{n}{Console}\PY{p}{.}\PY{n}{WriteLine}\PY{p}{(}\PY{l+s}{\PYZdq{}\PYZbs{}nStudent Details:\PYZdq{}}\PY{p}{)}\PY{p}{;}
\PY{n}{Console}\PY{p}{.}\PY{n}{WriteLine}\PY{p}{(}\PY{l+s}{\PYZdq{}Name: Mohit Tharu\PYZdq{}}\PY{p}{)}\PY{p}{;}
\PY{n}{Console}\PY{p}{.}\PY{n}{WriteLine}\PY{p}{(}\PY{l+s}{\PYZdq{}Roll No: 117\PYZdq{}}\PY{p}{)}\PY{p}{;}
\PY{n}{Console}\PY{p}{.}\PY{n}{WriteLine}\PY{p}{(}\PY{l+s}{\PYZdq{}College: Patan Multiple Campus\PYZdq{}}\PY{p}{)}\PY{p}{;}

\PY{n}{Console}\PY{p}{.}\PY{n}{ReadLine}\PY{p}{(}\PY{p}{)}\PY{p}{;}
\end{Verbatim}
\end{tcolorbox}

    \begin{Verbatim}[commandchars=\\\{\}]
Program started
Long task started{\ldots}
Doing other work{\ldots}
Long task completed.
Program ended

Student Details:
Name: Mohit Tharu
Roll No: 117
College: Patan Multiple Campus
    \end{Verbatim}

    \begin{tcolorbox}[breakable, size=fbox, boxrule=1pt, pad at break*=1mm,colback=cellbackground, colframe=cellborder]
\prompt{In}{incolor}{ }{\boxspacing}
\begin{Verbatim}[commandchars=\\\{\}]

\end{Verbatim}
\end{tcolorbox}


    % Add a bibliography block to the postdoc
    
    
    
\end{document}
